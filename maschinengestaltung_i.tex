\documentclass[a4paper,parskip=half*,DIV=7,fontsize=11pt]{scrartcl}
\usepackage[head=27.2pt]{geometry}
\usepackage[english,ngerman]{babel}
\usepackage[utf8x]{inputenc}
\usepackage{amsmath}
\usepackage{amssymb}
\usepackage{mathtools}
\usepackage{scrlayer-scrpage}
\usepackage{braket}
\usepackage{listings}
\usepackage{lastpage}
\usepackage{hyperref}
\usepackage{xcolor}
\usepackage{ragged2e} 
\usepackage{array}
\usepackage{rotating}

\lstset{
	mathescape=true,
	%    numbers=left
}

\lstset{literate=%
	{Ö}{{\"O}}1
	{Ä}{{\"A}}1
	{Ü}{{\"U}}1
	{ß}{{\ss}}1
	{ü}{{\"u}}1
	{ä}{{\"a}}1
	{ö}{{\"o}}1
	{~}{{\textasciitilde}}1
}

\ihead{MG I Panikzettel}
\title{MG I Panikzettel}
\author{Caspar Zecha}
\cfoot{\thepage\ / \pageref{LastPage}}

\lstset{basicstyle=\ttfamily}

\begin{document}
	
\maketitle
	
\begin{abstract}
	Dieser Panikzettel ist über die Vorlesung Maschinengestaltung I und basiert auf der Vorlesung von Univ.-Prof. Dr.-Ing. Georg Jacobs vom Institut für Allgemeine Konstruktionstechnik.\\
	Der aktuelle Master liegt $\href{http://panikzettel.philworld.de/}{hier}$.\\
	Möge eure Leistung in der Klausur reibungsfrei übertragen werden und das Ergebnis in eurer Toleranzzone liegen.\\
\end{abstract}
	
\tableofcontents
	
\pagebreak
	
\section{Einleitung}
\begin{itemize}
	\item Normen: DIN(National), EN(Regional), ISO(International)
	\item Geometrische Produktspezifikation: Geometrie ermöglicht Funktion, Tolerierung stellt Funktion sicher
\end{itemize}
	
\subsection{Dreitafelprojektion}
"Koordinatensystem" mit 4 Quadranten:
\begin{itemize}
	\item Oben Links: Vorderansicht
	\item Oben Rechts: Seitenansicht von links
	\item Unten Links: Draufsicht
	\item Linie in 45° zum Spiegeln
\end{itemize}
	
\subsection{Darstellung}
\begin{itemize}
	\item Dimetrische Darstellung: Ansicht mit 7°/42°, Kanten nach hinten nur 50 \%
	\item Isometrische Darstellung: Ansicht mit 30°/30°, alle Kanten 100 \%
\end{itemize}
	
\section{Elemente der technischen Zeichnung}
Maßstäbliche Darstellung, z.B. 2:1, Vergrößerung, oder 1:5, Verkleinerung.
	
\subsection{Liniengruppen}
\begin{itemize}
	\item 0,5: breite Volllinie = 0,5, schmale Volllinie = 0,25, Maß-, Textangaben = 0,35, Blattformate = A2, A3, A4
	\item 0,7: breite Volllinie = 0,7, schmale Volllinie = 0,35, Maß-, Textangaben = 0,5, Blattformate = A0, A1
\end{itemize}
	
\subsection{Linienarten}
\begin{itemize}
	\item Breite Volllinie: Körperkanten
	\item Schmale Volllinie: Bemaßung, Gewinde, Schraffur
	\item Freihandlinie: Unterbrochen dargestellte Schnittansicht, trifft Volllinie in 90°
	\item Schmale Strichpunktlinie: Symmetrielinie
	\item Breite Strichpunktlinie: Schnittebenen und Schnittverläufe
\end{itemize}
	
\section{Fertigungsgerechte Bemaßung}
\subsection{Allgemein}
\begin{itemize}
	\item Fertigungsgerechte Bemaßung bedeutet, dass alle Maße ohne Rechnung ablesbar sind.
	\item Lesbarkeit immer von unten oder rechts
	\item Maßlinien sind dünne Volllinien, mit ausgefüllten 15° Pfeilen an beiden Enden, dürfen nicht geschnitten werden
	\item die Maßangabe in $mm$ liegt auf der Maßlinie
	\item Maßhilfslinien sind ebenfalls dünne Volllinien und dürfen sich schneiden
	\item Die Pfeile sind je nach Liniengruppe 3,5 bis 5 mm lang
	\item Hinweißlinien die auf einer Fläche enden, enden mit einem Punkt
	\item Kettenbemaßung ist nicht zulässig
	\item Maßzahlen müssen frei stehen, ggf. Schraffur unterbrechen
\end{itemize}
	
\subsection{Bauteile}
\begin{itemize}
	\item Runde Bauteile, wie Wellen, mit dem Durchmessersymbol vor der Maßangabe
	\item Gewindebemaßung mit M, Radien mit R, Kugeln mit S jeweils vor der Maßangabe
	\item Fasen nur bei 45° mit Maßangabe: z.B. 2x45°, sonst Winkel und Länge
	\item Neigungssymbol und Kegelverjüngung in Richtung des Gefälle
\end{itemize}

\subsection{Vorgehen}
\begin{itemize}
	\item Wellen: Bemaßung von beiden Seiten + Durchmesser + Gesamtlänge
	\item Bleche: Bemaßung von zwei nicht gegenüberliegenden Seiten + Tiefe
\end{itemize}
	
	
\section{Schnitt- und Bruchdarstellungen}
\subsection{Allgemein}
\begin{itemize}
	\item Schnitte stellen innere Besonderheiten des Bauteils dar
	\item dünne Schraffur von Schnittflächen in 45° bzw. 135°, angrenzende Flächen verschieden schraffieren, zusammengehörende Flächen aber gleich
	\item Bruchkanten durch schmale Freihandlinie
\end{itemize}
		
\subsection{Schnittarten}
\begin{itemize}
	\item Halbschnitt: Darstellung eines symmetrischen Körpers zur unteren Hälfte als Schnitt, obere Hälfte ganz; horizontal liegt der geschnittene Bereich rechts
	\item Vollschnitt: Schnitt durch das gesamte Bauteil in der gekennzeichneten Ebene
	\item Teilschnitt: Freihandlinie um z.B. Wellennuten darzustellen
	\item Teilausschnitt: Darstellung eines Teilbereichs ohne Begrenzung
	\item Profilschnitt: Geschnittene Ansicht ins Bauteil gedreht
	\item Einzelheiten: Stelle mit Kreis und Buchstaben(z.B. X) kennzeichnen, und an anderer Vergrößerung in Teilausschnitt mit Vergrößerungsverhältnis darstellen 
\end{itemize}

\subsection{Schnittverlauf}
\begin{itemize}
	\item Dicke Strichpunktlinie an Anfang, Ende und allen Knicken des Verlaufs
	\item Bei mehreren Schnitten mit Buchstaben bezeichnen
	\item Blickrichtung: Dicke 30° Pfeile an Anfang und Ende des Verlaufs
\end{itemize}
		
\subsection{Ungeschnittene Bauteile}
\begin{itemize}
	\item Schrauben und Muttern
	\item Scheiben, Nieten und Stifte
	\item Bolzen, Federn und Keile
	\item Wälzkörper
	\item Teile ohne verdeckte Elemente oder Hohlräume
	\item Auch Elemente die sich vom Körperprofil abheben:\\
	      Rippen, Stege, Speichen, Wellen und Achsen
\end{itemize}
	
\section{Gewinde und Schraubenverbindungen}
\begin{itemize}
	\item Befestigen Bauteile durch Klemmkraft
	\item Schrauben sind selbst hemmend, lösen sich also nicht durch Zugkräfte
	\item Nehmen normal keine Querkräfte auf, außer Paßschraube
	\item Gewinde verwandeln rotatorische in translatorische Bewegung um
	\item Standard: Rechtsgewinde, sonst Kennzeichnung mit LH für Linksgewinde
	\item Regel: Innengewinde vor Außengewinde beim Zeichnen!
\end{itemize}
		
\subsection{Schraubenköpfe}
\begin{itemize}
	\item Sechskant, ggf. mit Telleransatz
	\item Zylinderschraube mit Innensechskant
	\item Senkschraube mit Innensechskant
	\item (Linsen-)Zylinderkopf
	\item (Linsen-)Senkkopf
\end{itemize}
	
\subsection{Schraubenschäfte}
\begin{itemize}
	\item Vollschaftschraube: Schaftdurchmesser =
	      Gewindedurchmesser 
	\item Dünnschaftschraube: Schaftdurchmesser =
	      Flankendurchmesser 
	\item Paßschraube: Schaftdurchmesser $>$ Gewindedurchmesser 
\end{itemize}
	
\subsection{Schraubenenden}
\begin{itemize}
	\item ohne Kuppe
	\item Spitze (120° im Normalfall)
	\item Linsen-/Kegelkuppe
	\item Zapfen
	\item Ringschneide
\end{itemize}
	
\subsection{Mutternarten}
\begin{itemize}
	\item (Flache) Sechskantmutter
	\item Vierkantmutter
	\item Kronenmutter
	\item Hutmutter
\end{itemize}
	
\subsection{Gewindegeometrie}
\begin{itemize}
	\item Teilung: Abstand von zwei Flanken
	\item Steigung: Höhengewinn bei einer vollen Umdrehung, relevant bei mehreren unabhängigen Flanken
	\item Flankenwinkel: Winkel zwischen zwei Flanken
	\item Kerndurchmesser: Durchmesser ohne Gewinde
	\item Außengewinde: Gesamtdurchmesser
	\item Schnittschraffur: Bis zur Gewindelinie, ggf. über äußere Gewindelinie
\end{itemize}
	
\subsection{Gewindeprofile}
\begin{itemize}
	\item Spitzgewinde: Standard für Schraube und Muttern
	\item Trapezgewinde: Bewegungs- und Verstellspindeln
	\item Sägegewinde: Spindeln mit einseitig hoher Belastung
	\item Rundgewinde: Spindeln mit hoher Abnutzung
\end{itemize}
	
\subsection{Außengewinde}
\begin{itemize}
	\item Gewindelänge: Nutzbare Länge mit Gewinde, einschließlich Kuppe
	\item Kegelkuppe: Ende der Schraube mit kleiner werdendem Gewinde
	\item Gewindeabschlusslinie(Volllinie): Zwischen Gewindeende und Schaft, ggf. Gewindeauslauf
	\item Sicht auf Stirnfläche: $\frac{3}{4}$-tel Kreis mit Kerndurchmesser
	\item Gewindefreistich: innerhalb der Gewindelänge
\end{itemize}
	
\subsection{Innengewinde}
\begin{itemize}
	\item Frontalsicht: Innendurchmesser = Volllinie, Außen dünner $\frac{3}{4}$-tel Kreis
	\item Sacklochbohrung: Tiefer als Gewindelänge bohren
\end{itemize}
	
\subsection{Schraubensicherung}
\begin{itemize}
	\item Spannscheibe: Baut durch axiale Stauchung Kraft auf
	\item Formschlüssige Verliersicherung: Kronenmutte+Spint, Drahtsicherung, Sicherungsblech
	\item Kraftschlüssige Verliersicherung: Federringe und -scheiben
\end{itemize}
	
\section{Welle-Nabe-Verbindung}
\begin{itemize}
	\item Stiftverbindungen: Nehmen Scherkräfte auf, sichern Lage aneinander liegender Teile, z.B. bei Naben, neben Schrauben oder als Steckstifte
	\item Bolzenverbindungen: Gelenkverbindung mit einem Freiheitsgrad, ggf. mit Kopf
	\item Keilverbindungen: vorgespannte Welle-Nabe-Verbindung, Wirkung durch Formschluß und Reibung
	\item Pass- und Scheibenfeder: für konstante Lasten, Kräfte nur an seitlichen Flächen übertragen
	\item Zahn- und Keilwellenverbindung: Übertragen von Drehmomenten, nur Profilierung von Nabe und Welle; Innenzentrierung: besserer Rundlauf, Flankenzentrierung: kleines Verdrehspiel und bessere Momentübertragung
	\item Prinzip von Druckhülse(ggf. mit Medium) und Sternscheibe: Druckkraft $\rightarrow$ Querkraft
\end{itemize}	
	
\section{Lagerung von Wellen}
\subsection{Fest-Los-Lagerung}
\begin{itemize}
	\item Festlager fixiert Welle in radialer und axialer Richtung, Loslager nur in radialer
	\item Zweck: Ausgleich der Wellenausdehnung
\end{itemize}

	
\subsection{Stütz-Traglagerung}
\begin{itemize}
	\item beide Lager nehmen Kräfte in je eine Richtung auf
	\item nur bei kurzen Wellen
	\item O- oder X-Anordnung, je nach Kraftfluss
\end{itemize}
	
\subsubsection{Schwimmende Lagerung}
Stütz-Traglagerung mit axialem Spiel:
Keine eindeutige Lagerung, aber unempfindlich und günstig
	
\subsection{Lagerungsprinzipien}
\begin{itemize}
	\item Gleitreibung durch Flüssigkeit
	\item Rollreibung durch Wälzkörper
\end{itemize}
	
\subsection{Wälzlager}
\begin{itemize}
	\item Käfig hält Wälzkörper seitlich in Lage
	\item Wälzkörper: Kegel, Kugel, Nadel
	\item Innenring, Außenring als Anlageflächen
\end{itemize}
	
\subsection{Kugellager}
\begin{itemize}
	\item Rillenkugellager (ein-/zweireihig)
	\item Schulter-/Schrägkugellager (ein-/zweireihig)		
\end{itemize}
	
\subsection{Rollenlager}
\begin{itemize}
	\item Zylinder-/Kegelrollenlager (ein-/zweireihig)
	\item Nadellager		
\end{itemize}
	
\subsection{Axiale Sicherung von Wälzlagern}
\begin{itemize}
	\item Mutter und Sicherungsscheibe
	\item Sicherungsring
	\item Endscheibe
\end{itemize}
	
\subsection{Sicherungsringe}
\begin{itemize}
	\item Sichern gegen Verschieben
	\item Können Kräfte entlang der Welle aufnehmen
	\item für Wellen und Bohrungen 
\end{itemize}
	
\subsection{Umlaufverhältnisse}
\begin{tabular}{|c|p{5cm}|p{5cm}|}
	\hline
	Belastung $\downarrow$ \ Umlaufender Ring $\rightarrow$ & Innenring                                                                              & Außenring                                                                             \\
	\hline
	Unveränderliche Richtung                               & Umfangslast Innen + Punktlast Außen  $\rightarrow$  Feste Passung Innen + Lose Außen & Punktlast Innen + Umfangslast Außen  $\rightarrow$  Lose Passung Innen + Feste Außen \\
	\hline
	Umlaufender Ring                                        & Punktlast Innen + Umfangslast Außen  $\rightarrow$ Lose Passung Innen + Feste Außen  & Umfangslast Innen + Punktlast Außen  $\rightarrow$ Feste Passung Innen + Lose Außen  \\
	\hline
\end{tabular}
	
\subsection{Dichtungen}
\subsubsection{Dynamische Dichtungen}
\begin{itemize}
	\item Labyrinthdichtung
	\item Dichtung mit Flüssigkeitssperrung 
\end{itemize}
	
\subsubsection{O-Ring}
\begin{itemize}
	\item Einbau in Rechtecknut
	\item Darstellung unter Druck gequetscht
\end{itemize}
	
\subsubsection{Radialwellendichtring}
\begin{itemize}
	\item Versteifungsring aus Metall für Stabilität
	\item Außenmantel und Schutzlippe zum Abdichten
	\item Zugfeder innen
\end{itemize}
	
\section{Leistungsübertragung}
Einteilung in gleichmäßig und ungleichmäßig übersetzende Getriebe
\subsection{Zugmittelgetriebe}
\begin{itemize}
	\item Reibschlüssig: Riemenscheibe mit Flach-, Rund oder Keilriemen
	\item Formschlüssig: Hülsenkette auf Kettenrad, Zahnkette auf Zahnrad und Synchronriemen auf Synchronscheibe
\end{itemize}
	
\subsection{Zahnradgetriebe}
\begin{itemize}
	\item zwischen zwei oder mehr parallelen oder kreuzenden Wellen
	\item geringe Verluste
	\item kleines Zahnrad: Ritzel (antreibend), großes Zahnrad: Rad (angetrieben)
	\item Schrägverzahnung: Laufruhe, aber teurer und benötigt besseres Lager
	\item Pfeilverzahnung: Mit schwimmender Lagerung
	\item Zähnezahl $z$, Modul $m$, Teilkreisdurchmesser $d= m \cdot z$
	\item Formeln: $i=\frac{\omega_1}{\omega_2}, i_{ges}=\prod_{j}{} i_j, M_{ein} \cdot i_{ges}=M_{aus}$
\end{itemize}
	
\subsubsection{Arten}
\begin{itemize}
	\item Stirnradgetriebe: Zähnezahl von Ritzel und Rad teilerfremd
	\item Kegelradgetriebe: Achsen der Kegelräder schneiden sich
	\item 	Schneckengetriebe: Kreuzende Achsen; meist selbst hemmend; Achsen liegen nicht in einer Ebene
	\item 	Planetengetriebe: mehrstufige Stirnrädergetriebe, große Übersetzungsverhältnisse; geringe Belastung
\end{itemize}	
	
\section{Maßtoleranzen und Passungen}
\subsection{Toleranzen}
\begin{itemize}
	\item Direktes Antragen ans Maß mittels kleiner Zahlen oben und unten rechts
	\item Allgemeintoleranzen: Toleranzklassen enthalten Toleranzen für viele Maße, gelten nicht für bereits tolerierte Elemente.
	\item ISO-Toleranzfelder: Toleranzintervalle mit Toleranztabelle
\end{itemize}
	
\subsection{Passungen}
\begin{itemize}
	\item Spiel: Positive Differenz zwischen Bohrung und Welle
	\item Übermaß: Negative Differenz vor dem Fügen
	\item Spielpassung: Mindestmaß der Bohrung $\geq$ Höchstmaß der Welle
	\item Übergangspassung: Beim Fügen entsteht Spiel oder Übermaß
	\item Übermaßpassung: Höchstmaß der Bohrung $\leq$ Mindestmaß der Welle
	\item Passtoleranz: Betragsmäßige Summe der Toleranzen von Bohrung und Welle
\end{itemize}
Einführung von Einheitsbohrung und Einheitswelle, mit Toleranzfeldlage H/h, um Kosten zu sparen. 
	
Bei gegebenem Durchmesser, sowie den gewünschten Spielen/Übermaßen: Auswahl der Passung mithilfe der Passungsauwahl DIN 7157.
	
\section{Form- und Lagetoleranzen}
Es gibt Gestaltabweichung in Maß, Form, Lage und Oberfläche.
	
Allgemeintoleranzen tolerieren mit Toleranzklassen nicht alle Eigenschaften, z.B. Koaxialität nicht.
\subsection{Grundsätze}
\begin{itemize}
	\item Unabhängigkeit: Maß- und Formtoleranz können jeweils unabhängig ihr Maximum erreichen
	\item Hüllbedingung: Maßtoleranzen begrenzen Toleranzzone der Form
	\item Ausnahmen z.B. durch $\textcircled{\tiny E}$ hinter der Maßangabe
\end{itemize}
	
\subsection{Zeichnungseintragung}
\begin{itemize}
	\item Rechteckiger Rahmen mit 15° Pfeil senkrecht auf zu tolerierendes Bauteil
	\item Mehrere Felder im Rechteck:
	      \begin{enumerate}
	      	\item Feld: Symbol für toleriertes Merkmal
	      	\item Feld: Toleranzwert, ggf. mit Durchmessersymbol, oder $S$ für Kugeln
	      	\item und ggf. weitere Felder: Buchstabe als Bezug
	      \end{enumerate}
	\item Bezug auf Teil mit ausgefüllter Pyramide und dünner Volllinie auf Rechteck mit Großbuchstaben
\end{itemize}

\subsection{Toleriertes Element}
\begin{tabular}{|c|c|}
	\hline
	\textbf{Hinweislinie zeigt auf:} & \textbf{Toleriertes Geometrieelement}\\
	\hline
	Zylinder, aber nicht auf Maßlinie & Teil des Zylinders\\
	\hline
	Verlängerung der Maßlinie des Zylinders & Teil der Achse des Zylinders\\
	\hline
	Ebene, aber nicht auf Maßlinie & Teil der Ebene\\
	\hline
	Maßlinie zwischen zwei entgegengesetzt & Teil der Mittelebene von zwei Ebenen \\ gerichteten parallelen Ebenen & \\
	\hline
\end{tabular}
	
\subsection{Formtoleranzen}
Tolerierte Elemente:
\begin{itemize}
	\item Geradheit
	\item Ebenheit
	\item Rundheit
	\item Zylinderform
\end{itemize}
Jeder Punkt muss sich innerhalb der Toleranzzone befinden. Die Lage relativ zum Nennmaß ist nicht vorgegeben.
	
\subsection{Lagetoleranzen}
Tolerierte Elemente:
\begin{itemize}
	\item Richtung: Parallelität, Rechtwinkligkeit, Neigung
	\item Ort: Position, Koaxialität, Symmetrie
	\item Lauf: Plan-/Rundlauftoleranz
\end{itemize}
Jeder Punkt muss sich innerhalb der Toleranzzone befinden. Nur in Bezug auf andere Geometrieelemente.
	
\section{Technische Oberflächen und Kanten}
\subsection{Ordnungssystem für Gestaltabweichungen}
\begin{enumerate}
	\item Ordnung: Formabweichungen(Geradheit, Ebenheit, Rundheit)
	\item Ordnung: Welligkeit(Wellen)
	\item Ordnung: Rauheit(Rillen)
	\item Ordnung: Rauheit(Riefen, Schuppen, Kuppen)
\end{enumerate}

Ist-Oberfläche: Überlagerung von 1. bis 4. Ordnung

\subsection{Oberflächenkenngrößen}
\begin{itemize}
	\item Rz: Gemittelte Rautiefe aus 5 Messstrecken 
	\item Ra: Mittenrauwert
\end{itemize}
	
\subsection{Oberflächensymbole}
\begin{itemize}
	\item Grundsymbol: kein vorgeschriebenes Fertigungsverfahren
	\item mit geschlossenem Dreieck: Materialabtrennendes Verfahren
	\item mit Kreis unten: Kein Materialabtrennendes Verfahren
	\item mit Kreis oben: Gleiche Oberflächenbeschaffenheit für alle Flächen eines Teils
	\item Vereinfachte Eintragung mit $x$ oder $y$ und Erklärung an anderer Stelle
	\item Vereinfachte Legende mit normaler Beschaffenheit und ggf. leerer Klammer für explizit eingetragene Werte
\end{itemize}
	
\subsection{Darstellung von Kanten}
\begin{tabular}{|c|c|c|}
	\hline
	Außenkante & Innenkante & Zeicheneintragungssymbol \\
	\hline
	gratig      & Übergang  & +                        \\
	\hline
	gratfrei    & Abtragung  & -                        \\
	\hline 
\end{tabular}

Darstellung in der Zeichnung mit Hinweispfeil und -linie auf Kante, sowie Innen-/Außenkante und +/- Wert, z.B. +0,1.
	
\section{Schweißen}
Schweißen ist das Vereinigen von Werkstoffen in der Schweißzone unter
Anwendung von Wärme und/oder Kraft ohne oder mit Schweißzusatz.
\begin{itemize}
	\item Verbindungsschweißen: Zusammenfügen von Teilen mit Schweißnähten am Schweißstoß zum Schweißteil
	\item Schweißgruppe: Mehrere Schweißteile ergeben die Schweißgruppe
	\item Schweißkonstruktion: Besteht aus mehreren Schweißkonstruktionen
\end{itemize}

\subsection{Schweißverfahren}
\subsubsection{MIG-/MAG-Schweißen}
\begin{itemize}
	\item Metall-Schutzgas-Schweißen
	\item hohe Abschmelzleistung und Schweißgeschwindigkeit
	\item gut automatisierbar
	\item Nachteil Wärme: Anfangsbindefehler und Endkraterrisse
\end{itemize}
	
\subsubsection{WIG-Schweißen}
\begin{itemize}
	\item Trennung von Wärme und Zusatzwerkstoff: Schmelzbad besser beeinflussbar
	\item In vielen Schweißposition nutzbar, z.B. Reparatur
	\item hochwertige Schweißverbindungen
	\item geringe Leistung ud Geschwindigkeit
	\item schwierige Automation
\end{itemize}

\subsubsection{Laserstrahlschwißen}
\begin{itemize}
	\item wenig Wärme; hohe Leistung und Geschwindigkeit
	\item Präzise und gut automatisierbar
	\item Teurer als andere Verfahren
\end{itemize}

\subsection{Stoßarten}
\begin{itemize}
	\item Stumpfstoß: \textbf{- -}
	\item Parallelstoß: \textbf{=}
	\item T-Stoß: \textbf{T}
	\item Kreuzstoß: \textbf{-$\mid$-}
	\item Eckstoß: \textbf{L}
	\item Überlappungsstoß, Mehrfachstoß, Schrägstoß
\end{itemize}
	
\subsection{Nähte}
\begin{itemize}
	\item V-Naht: \textbf{V}
	\item Kehlnaht
\end{itemize}
	
\subsection{Zeichnungseintragung}
\begin{itemize}
	\item dünne Pfeillinie mit 15° Pfeil auf Fügekante
	\item Bezugslinie und ggf. Bezugslinie-Strichlinie für Gegenseite untereinander an Pfeillinie
	\item Nahtzeichen auf gewünschte Linie setzen
	\item Doppelkehlnaht: Zeichen auf und unter Bezugslinie anstatt Strichlinie
	\item Kreis an Treffpunkt von Bezugs- und Pfeillinie: Umlaufende Naht
	\item Bemaßung mit $a$, Dicke der Naht, Symbol und Länge, z.B.: $a4V20$
\end{itemize}
Die beste Schweißnaht ist keine Schweißnaht, wegen Veränderung durch Wärme an Steifigkeit und Festigkeit.

\pagebreak
\section{Lösungswege für Aufgaben}
Im folgenden werden Tipps und Vorgehensweisen für bestimmte Aufgabentypen gegeben:
\subsection{Dreitafelprojektion}
\begin{enumerate}
	\item Zeichnung und gegebene Ansicht solange hart anstarren, bis ein drehbares Modell im Kopf ist
	\item Auf die Blickrichtung der schon gegebenen Ansicht achten und das Modell entsprechend drehen
	\item Mithilfe der Spiegellinie und der gegebenen Ansicht die restlichen Ansichten herleiten
	\item Zuerst die Außenkanten zeichnen und dann ins Detail arbeiten
\end{enumerate}

\subsection{Bemaßung}
\subsubsection{Bleche}
\begin{itemize}
	\item Von zwei nicht gegenüberliegenden Seiten alle Maße antragen
	\item Dicke angeben, ggf. Wert ausdenken
	\item Bohrungen, Fasen, Radien etc. mit Symbol und ggf. Lage vom Rand bemaßen
\end{itemize}

\subsubsection{Wellen}
\begin{itemize}
	\item Von beiden Seiten bis zum dicksten Absatz bemaßen und Gesamtlänge angeben
	\item Bohrungen, Fasen etc. mit Symbol und ggf. Lage vom Rand bemaßen
\end{itemize}

\subsection{Schnittdarstellung}
\begin{itemize}
	\item An einer Seite anfangen und zur anderen durcharbeiten
	\item Geschnittene Flächen dünn schraffieren
	\item Gewinde mit Symmetrielinien einzeichnen
	\item An nicht geschnittene Bauteile denken, falls nicht anders angegeben
	\item ggf. Schnittverlauf und Blickrichtung in gegebener Ansicht einzeichnen
\end{itemize}

\subsection{Schrauben und Gewinde}
\begin{itemize}
	\item Außengewinde vor Innengewinde
	\item Bohrungen tiefer als Gewinde und mit 120° Spitze, sowie Gewinde länger als Schraubenlänge
\end{itemize}

\subsection{Passungen}
\begin{itemize}
	\item Tabellen mitnehmen, Zahlen mit Starren heraussuchen und ausrechnen
	\item Passung gesucht: Passung mit größtmöglichem Bereich wählen
	\item ggf. auf Kosten achten, z.B. h6 Einheitswelle
\end{itemize}

\subsection{Welle-Nabe-Verbindung}
\begin{itemize}
	\item Freihandlinie für Darstellung der Nut mit Schraffur
	\item Freihandlinie trifft Kante mit 90°
\end{itemize}

\subsection{Lagerung von Wellen}
\begin{itemize}
	\item Lagerart und ggf. Anordnung erkennen
	\item Meist nur mit Sicherungsringen und Wellenabsätzen befestigen
	\item Bei Deckeln etc. auf Doppelpassung achten
\end{itemize}

\subsection{Leistungsübertragung}
	Formeln %TO-DO
	
\subsection{Form- und Lagetoleranzen, Oberflächen und kanten}
\begin{itemize}
	\item Zeichen und Toleranzen zuordnen können
	\item Mit Kästchen nach und nach die geforderten Toleranzen anhand der IT Qualität einzeichnen
	\item Oberflächen meist mit Ra und ggf. auf Verfahren achten
	\item sinnvolle Werte für Kanten: $\pm$0.1
\end{itemize}

\subsection{Schweißen}
noch keine Tipps %TO-DO

\subsection{Multiple Choice}
Panikzettel hart anstarren und auswendig können!
	
\end{document}