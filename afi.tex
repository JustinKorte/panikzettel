\documentclass{panikzettel}

\usepackage{accents}
\usepackage{tabularx}
\newcolumntype{Y}{>{\centering\arraybackslash}X}
\newcolumntype{M}{>{\centering\arraybackslash$\displaystyle}X<{$}}

\newcommand{\N}{\mathbb{N}}
\newcommand{\Q}{\mathbb{Q}}
\newcommand{\R}{\mathbb{R}}
\newcommand{\CC}{\mathbb{C}}
\renewcommand{\Re}{\mathrm{Re}}
\renewcommand{\Im}{\mathrm{Im}}

\title{Analysis für Informatiker Panikzettel}
\author{Philipp Schröer}

\begin{document}

\maketitle

\tableofcontents

\section{Einleitung}

Dieser Panikzettel ist über die Vorlesung Analysis für Informatiker bei Prof.\ Stamm im Wintersemester 2017/18.

Dieser Panikzettel ist Open Source.
Wir freuen uns über Anmerkungen und Verbesserungsvorschläge auf \url{https://git.rwth-aachen.de/philipp.schroer/panikzettel}.

\section{Grundlagen}

Wir werden hier \emph{nicht} die mathematischen Grundlagen der Vorlesung wiederholen. Induktion, Rekursion und Zahlen sollten schon verstanden sein. Die wichtigsten Ergebnisse wollen wir aber trotzdem hier sammeln.

\subsection{Formeln und Ungleichungen}

Das allgemeine Nullstellenproblem lässt sich über Formeln lösen:
\begin{alignat*}{3}
  ax^2 + bx + c = 0
  &\quad\Leftrightarrow\quad
  &&x = \frac{-b \pm \sqrt{b^2 - 4ac}}{2a} \\
  x^2 + px + q = 0
  &\quad\Leftrightarrow\quad
  &&x = -\frac{p}{2} \pm \sqrt{\left(\frac{p}{2}\right)^2 - q}
\end{alignat*}

Für den \emph{Binomialkoeffizienten} $\begin{pmatrix} n \\ k \end{pmatrix} := \frac{n!}{k! \cdot (n-k)!} \in \Q$ gilt mit $k,n \in \N_0$ und $k \leq n$:
\begin{align*}
  \begin{pmatrix} n \\ k \end{pmatrix} &= \begin{pmatrix} n \\ n - k \end{pmatrix} \\
  \begin{pmatrix} n \\ k - 1 \end{pmatrix} + \begin{pmatrix} n \\ k \end{pmatrix} &= \begin{pmatrix} n + 1 \\ k \end{pmatrix}, \text{ falls } k > 0 \\
  \begin{pmatrix} n \\ k \end{pmatrix} &\in \N
\end{align*}

Die \emph{Binomische Formel} für $n \in \N_0$ und $a,b \in \mathbb{C}$:
\begin{align*}
  (a + b)^n = \sum_{k=0}^n \begin{pmatrix} n \\ k \end{pmatrix} a^k b^{n-k} = \sum_{k=0}^n \begin{pmatrix} n \\ k \end{pmatrix} a^{n-k} b^k
\end{align*}

Die \emph{Geometrische Summenformel} ist praktisch. Mit $q \neq 1$ und $a \neq b$:
\begin{align*}
  \sum_{k=0}^n q^k &= 1 + q + \ldots + q^n = \frac{q^{n+1}-1}{q-1} = \frac{1-q^{n+1}}{1-q} \\
  \sum_{k=0}^n a^k b^{n-k} &= \frac{a^{n+1}-b^{n+1}}{a-b}
\end{align*}

Die \emph{Bernoullische Ungleichung} für $a \geq -1$:
\[
  (1+a)^n \geq 1 + na
\]

\subsection{Unendlichkeit}

Eine Menge $M$ ist genau dann \emph{abzählbar}, wenn eine injektive Abbildung $g : M \to \N$ existiert.

Wir können eine bijektive Abbildung $\phi : \N \to \N \times \N$ bauen, sodass auch $\N \times \N$ abzählbar ist. Also ist eine abzählbare Vereinigung abzählbarer Mengen auch abzählbar. $\Q$ ist abzählbar.

Dagegen ist $\text{Pot}(\N)$ überabzählbar, genau wie $\Set{0,1}^\N$, die Menge der Abbildungen von $\N$ nach $\Set{0,1}$.

\section{Folgen}
\label{sec:folgen}

Eine \emph{reelle Folge} ist eine Abbildung $\N \to \R$. Wir schreiben auch statt $\N \to \R, n \mapsto a_n$ kurz $(a_n)_{n \geq 1}$.

Eine Folge ist \emph{monoton wachsend}, wenn gilt $a_{n+1} \geq a_n$ für alle $n \in \N$. \emph{Monoton fallend} analog $a_{n+1} \leq a_n$. \emph{Streng monoton} mit jeweils $>$ bzw. $<$.

Eine Folge ist \emph{nach oben beschränkt}, wenn es ein $b \in \R$ gibt, sodass $a_n \leq b$ für alle $n \in \N$. \emph{Nach unten beschränkt} analog. Monoton wachsende Folgen sind nach unten, monoton fallende nach oben beschränkt. Eine Folge ist \emph{beschränkt}, wenn es ein $b \in \R$ gibt, sodass $\abs{a_n} \leq b$ für alle $n \in \N$.

Eine Folge heißt \emph{konvergent}, wenn ein $a \in \R$ existiert, so dass zu jedem $\varepsilon > 0$ ein $n_0$ (abhängig von $\varepsilon$) existiert mit
\[ \abs{a_n - a} < \varepsilon \text{ für alle } n \geq n_0 \]
Dann ist $a$ der \emph{Grenzwert} der Folge $(a_n)_{n \geq 1}$.
Eine \emph{Nullfolge} hat Grenzwert 0. Eine nicht konvergente Folge ist \emph{divergent}.

Eine (reelle) Folge hat höchstens einen Grenzwert. Eine konvergente Folge ist (durch den Grenzwert) beschränkt.

Die \emph{Limitenregeln} lassen Grenzwerte von konvergenten (!) Folgen zusammenrechnen.\\
Seien $(a_n)_{n \geq 1}, (b_n)_{n \geq 1}$ konvergente Folgen mit Grenzwerten $a$ respektive $b$:
\begin{alignat*}{4}
  \lim_{n \to \infty}\; && \alpha \cdot a_n &= \alpha \cdot a \\
  \lim_{n \to \infty}\; && a_n + b_n &= a + b \\
  \lim_{n \to \infty}\; && a_n \cdot b_n &= a \cdot b \\
  \lim_{n \to \infty}\; && \frac{a_n}{b_n} &= \frac{a}{b} &&\qquad \text{falls } b_n \neq 0\; \Forall n \in \N
\end{alignat*}

Die Folge $(a_n)_{n \geq 1}$ konvergiert genau dann, wenn $(a_n - a)_{n \geq 1}$ eine Nullfolge ist.

Das \emph{Sandwich-Lemma}: Für ein $n_0 \in \N$:
{\setlength{\abovedisplayskip}{0pt}
\[
  \begin{array}{c}
    \displaystyle
    \lim_{n \to \infty} a_n = \lim_{n \to \infty} b_n = b, \\
    a_n \leq b_n \leq c_n \quad\Forall n \geq n_0
  \end{array}
  \qquad\Longrightarrow\qquad
  \lim_{n \to \infty} b_n = b
\]}

Nützliche Aussagen zur Wurzelfunktion:
\begin{itemize}
  \item Für $q \in \R$ mit $\abs{q} < 1$ ist $(q^n)_{n \geq 1}$ eine Nullfolge.
  \item $\lim_{n \to \infty} \sqrt[n]{n} = 1$.
  \item Für $q \in \R$ mit $q > 0$ gilt $\lim_{n \to \infty} \sqrt[n]{q} = 1$.
\end{itemize}
\medskip

Der \emph{Satz von L'Hospital} erlaubt die Bestimmung von Grenzwerten über \hyperref[sec:ableitungen]{Ableitungen}.
Wenn $f$ und $g$ stetig differenzierbar sind und $\lim\limits_{x \to x_0} f(x) = \lim\limits_{x \to x_0} g(x) = 0$ oder $\lim\limits_{x \to x_0} f(x) = \lim\limits_{x \to x_0} g(x) = \infty$ gilt, dann folgt:
\[ \lim_{x \to x_0} \frac{f(x)}{g(x)} = \lim_{x \to x_0} \frac{f'(x)}{g'(x)} \]

Eine \emph{Teilfolge} $(a_k')_{k \geq 1}$ zur Folge $(a_n)_{n \geq 1}$ wird mit einer streng monoton wachsenden Indexfolge $(n_k)_{k \geq 1}$ gebaut. Jede Teilfolge konvergiert gegen den Grenzwert der ursprünglichen Folge.

Eine Folge heißt \emph{Cauchy-Folge}, wenn es zu jedem $\varepsilon > 0$ ein $n_0$ (abhängig von $\varepsilon$) gibt, so dass
\[ \abs{a_m - a_n} < \varepsilon \qquad \text{ für alle } m,n \geq n_0 \]
Eine Folge ist genau dann eine Cauchy-Folge, wenn sie konvergiert.

Eine Folge ist \emph{bestimmt divergent}, wenn es zu jedem $b \in \R, b > 0$ ein $n_0 \in \N$ gibt, so dass $a_n \geq b$ für alle $n \geq n_0$. Man schreibt $\lim_{n \to \infty} = \infty$. Analog für $-\infty$ mit $\leq$.

\section{Reihen}
\label{sec:reihen}

\subsection{Konvergenz}

Eine \emph{(unendliche) Reihe} besteht aus zwei Folgen: Der ``inneren Folge'' $(a_k)_{k \geq 1}$ und der Partialsummenfolge $(s_n)_{n \geq 1}$ sodass:
\[ s_n = \sum_{k=1}^n a_k \quad \text{für alle } n \in \N \]
Wir schreiben auch $\sum_{k=1}^\infty a_k$ für die Reihe und $\sum_{k=1}^\infty a_k = \lim_{n \to \infty} \sum_{k=1}^n a_k$.

Wir können konvergente Reihen zusammen und auseinander ziehen. Mit $\alpha, \beta \in \R$:
\[ \alpha \sum_{k=1}^\infty a_k + \beta \sum_{k=1}^\infty b_k = \sum_{k=1} (\alpha a_k + \beta b_k) \]

Das \emph{Cauchy-Kriterium für Reihen}: Die Reihe $\sum_{k=1}^\infty a_k$ konvergiert genau dann, wenn es zu jedem $\varepsilon > 0$ ein $n_0 \in \N$ gibt, sodass
\[ \abs{\sum_{k=m}^n a_k} < \varepsilon \quad \text{ für alle } n \geq m \geq n_0 \]

Wenn die Reihe $\sum_{k=1}^\infty a_k$ konvergiert, muss auch $(a_k)_{k \geq 1}$ eine Nullfolge sein.

Eine Reihe konvergiert genau dann, wenn die Folge der Partialsummen nach oben beschränkt ist.

Das \emph{Leibniz-Kriterium} hilft bei alternierenden Folgen. Sei $(a_k)_{k \geq 1}$ eine monoton fallende Nullfolge. Dann konvergiert $\sum_{k=1}^\infty (-1)^k a_k$ und es gilt für alle $n \in \N$:
\[ \abs{\sum_{k=n}^\infty (-1)^k a_k} \leq a_n \]

\subsection{Absolute Konvergenz}

Stärker als normale Konvergenz ist die \emph{absolute Konvergenz}: Hier muss für $\sum_{k=1}^\infty a_k$ sogar $\sum_{k=1}^\infty \abs{a_k}$ konvergieren.
\begin{itemize}
  \item Eine Reihe, die konvergiert, aber nicht absolut konvergiert, heißt \emph{bedingt konvergent}.
  \item Wenn eine Reihe absolut konvergent ist, ist sie auch konvergent.
\end{itemize}

Zum Zeigen der absoluten Konvergenz hilft das \emph{Majorantenkriterium}. Sei $\sum_{k=1}^\infty c_k$ konvergent. Wenn $\abs{a_k} \leq c_k$ für alle $k \geq n_0$ gilt, ist $\sum_{k=1}^\infty$ absolut konvergent. Das Gegenstück, das \emph{Minorantenkriterium}, zeigt Divergenz von $\sum_{k=1}^\infty \abs{a_k}$, wenn $\sum_{k=1}^\infty c_k$ divergent ist.

Alternativ das \emph{Quotientenkriterium}: Wenn $a_k \neq 0$ und $\abs{\frac{a_{k+1}}{a_k}} < 1$ für alle $k \geq n_0$ gilt, dann ist $\sum_{k=1}^\infty a_k$ absolut konvergent. Es geht auch: $\lim_{k \to \infty} \abs{\frac{a_{k+1}}{a_k}} < 1$. Wenn $> 1$, so folgt Divergenz. Achtung: $= 1$ sagt nichts aus!

\subsection{Die Exponentialfunktion}

Man kann die Eulersche Zahl $e$ ($e = e^1 = exp(1)$) durch eine Reihe darstellen.
Für $x \in \N$:
\[ e^x = \exp(x) := \sum_{k=0}^\infty \frac{x^k}{k!} \]

$\exp(x)$ für $x \in \Q$ ist über die üblichen Potenzgesetze berechenbar: $\exp(\frac{p}{q}) = \sqrt[q](\exp(p))$.

\subsection{Grenzwerttabelle}

\begin{alignat*}{5}
\text{\emph{Geometrische Reihe}}\qquad\qquad & \sum_{k=0}^\infty & q^k &= \frac{1}{1-q} && \qquad\text{für } \abs{q} < 1 \\
\text{\emph{Harmonische Reihe}}\qquad\qquad & \sum_{k=1}^\infty & \frac{1}{k} &= \infty \\
\text{\emph{``Huh?''}}\qquad\qquad & \sum_{k=1}^\infty & \frac{1}{k^2} &= \frac{\pi^2}{6}
\end{alignat*}

\section{Komplexe Zahlen}

\subsection{Definition und Rechenregeln}

Die \emph{komplexen Zahlen} sind definiert als $\CC = \Set{ z = x + iy | x,y \in \R }$.

Für $z = x + iy \in \CC$ definieren wir:
\begin{align*}
  \Re(z) := x && \Im(z) := y && \overline{z} := x - iy
\end{align*}

Rechenregeln:
\begin{align*}
  i^2 &= -1 \\
  (x + iy) + (u + iv) &= (x+u) + i(y+v) \\
  (x+iy) \cdot (u + iv) &= (xu - yv) + i(xv + yu)
\end{align*}

Inverse lassen sich direkt bestimmen:
\[ \frac{1}{z} = \frac{x}{x^2+y^2} - i \cdot \frac{y}{x^2+y^2} \]

Für die komplexe Konjugation gibt es auch einige weitere Rechenregeln.
\[ x = \frac{1}{2} (z + \overline{z}) \qquad\qquad y = \frac{1}{2i}(z-\overline{z}) \]
\[ \overline{z+v} = \overline{z} + \overline{w} \qquad \overline{z \cdot w} = \overline{z} \cdot \overline{w} \qquad \overline{\left( \frac{z}{w} \right)} = \frac{\overline{z}}{\overline{w}} \]
\[ z \cdot \overline{z} = x^2 + y^2 \]

Der \emph{Betrag} einer komplexen Zahl: $\abs{z} := \sqrt{z \cdot \overline{z}} = \sqrt{x^2 + y^2}$.
Für den Betrag gelten die üblichen Dreiecksungleichungen.

Nach dem \emph{Fundamentalsatz der Algebra} hat jedes Polynom in $\CC$ auch eine Lösung in $\CC$.

\subsection{Komplexe Folgen und Reihen}

Die Definitionen von \hyperref[sec:folgen]{Abschnitt 2} und \hyperref[sec:reihen]{Abschnitt 3} lassen sich einfach mithilfe des Betrages für komplexe Zahlen übertragen.
Einige Rechenregeln kann man auch übernehmen.

Wird die Definition der Exponentialfunktion auf $\CC$ erweitert, so wird es interessant.

\[ \exp(z + w) = \exp(z) \cdot \exp(w) \]
\[ \exp(\overline{z}) = \overline{\exp(z)} \]
\[ \abs{\exp(ix)} = 1 \qquad\Forall x \in \R \]

\[ \cos(x) := \Re(\exp(ix)) \qquad\qquad \sin(x) := \Im(\exp(ix)) \]

Per Definition gilt die \emph{Euler-Identität}:
\[ \exp(ix) = \cos(x) + i \cdot \sin(x) \]

Einige Rechenregeln für Sinus und Cosinus, mit $x,y \in \R$: \\
\begin{halfboxl}
  \vspace{-\baselineskip}
  \begin{align*}
    \cos(x+y) &= \cos(x) \cdot \cos(y) - \sin(x) \cdot \sin(y) \\
    \sin(x+y) &= \cos(x) \cdot \sin(y) + \sin(x) \cdot \cos(y) \\
    \cos(2x) &= \cos(x)^2 - \sin(x)^2 \\
    \sin(2x) &= 2 \sin(x) \cos(x) \\
    1 &= \cos(x)^2 + \sin(x)^2
  \end{align*}
\end{halfboxl}%
\begin{halfboxr}
  \vspace{-\baselineskip}
  \begin{align*}
    \sin(x) &= \cos(x - \frac{1}{2} \pi) \\
    \sin(x + \pi) = -\sin(x), & \quad \sin(x + 2 \pi) = \sin(x) \\
    \sin(2k \cdot \pi) = 0, & \quad \cos((2k+1) \cdot \frac{\pi}{2}) = 0 \\
    \tan(x) &= \frac{\sin(x)}{\cos(x)}
  \end{align*}
\end{halfboxr}

Zur Berechnung von Sinus und Cosinus:

{\setlength{\extrarowheight}{10pt}
\begin{tabularx}{\textwidth}{Y|Y|Y|Y|Y|Y|Y|Y|Y|Y}
  x & $0$ & $\frac{\pi}{6}$ & $\frac{\pi}{4}$ & $\frac{\pi}{3}$ & $\frac{\pi}{2}$ & $\frac{2\pi}{3}$ & $\frac{3\pi}{4}$ & $\frac{5\pi}{6}$ & $\pi$ \\
  $\sin(x)$ & $0$ & $\frac{1}{2}$ & $\frac{1}{2} \sqrt{2}$ & $\frac{1}{2} \sqrt{3}$ & $1$ & $\frac{1}{2} \sqrt{3}$ & $\frac{1}{2} \sqrt{2}$ & $\frac{1}{2}$ & $0$ \\
  $\cos(x)$ & $1$ & $\frac{1}{2} \sqrt{3}$ & $\frac{1}{2} \sqrt{2}$ & $\frac{1}{2}$ & $0$ & $-\frac{1}{2}$ & $-\frac{1}{2} \sqrt{2}$ & $-\frac{1}{2} \sqrt{3}$ & $-1$
\end{tabularx}}
\medskip

\section{Reelle Funktionen}

\subsection{Umgebungen}

Eine \emph{$\varepsilon$-Umgebung} um $x_0 \in \R$ mit $\varepsilon > 0$:
\begin{align*}
  U_\varepsilon(x_0) &:= \Set{ x \in \R | \abs{x-x_0} < \varepsilon } \\
                     &\hphantom{:}= (x_0 - \varepsilon,~ x_0 + \varepsilon)
\end{align*}

Ein \emph{innerer Punkt} $x_0 \in M$ von $M \subset \R$ ist ein $x_0$ so, dass ein $\varepsilon > 0$ existiert mit $U_\varepsilon(x_0) \subset M$.
Außerdem sei das \emph{Innere von M}, $\mathring{M}$, definiert als die Menge der inneren Punkte von M.

$M$ ist \emph{offen}, wenn alle $x \in M$ auch innere Punkte sind ($M = \mathring{M}$).
$M$ ist \emph{abgeschlossen}, wenn $\R \setminus M$ offen ist.

\subsection{Funktionen}

Der \emph{Graph} einer Funktion $f : A \to B$ ist die Menge $G_f = \Set{ (x, f(x)) | x \in A }$.

Wir definieren außerdem Addition und Multiplikation für Funktionen.\footnote{Mehr zu Funktionen als Vektorräume im \href{./la.pdf}{Panikzettel Lineare Algebra}.}
Sei $D \subset \R$ nicht leer und $f, g : D \to \R$ und $\alpha \in \R$.
\begin{alignat*}{3}
  f + g,\quad& x \to f(x) + g(x) \\
  \alpha f,\quad& x \to \alpha \cdot (f(x)) \\
  f \cdot g,\quad& x \to f(x) \cdot g(x) \\
  \frac{f}{g},\quad& x \to \frac{f(x)}{g(x)} &&\qquad \text{falls } g(x) \neq 0~~\Forall x \in D
\end{alignat*}

Monotonie lässt sich auf Funktionen übertragen.
$f : D \to E$ mit $D, E \subset \R$ ist \emph{monoton wachsend}, wenn für alle $x, y \in D$ mit $x < y$ immer $f(x) \leq f(y)$ gilt. \emph{Monoton fallend} analog mit $\geq$. \emph{Streng monoton wachsend} bzw. \emph{fallend} analog mit $<$ bzw. $>$.

Wenn $f$ streng monoton wachsend/fallend ist, dann ist $f^{-1}$ injektiv und ebenfalls streng monoton wachsend/fallend.

$f$ ist \emph{nach oben/unten beschränkt}, wenn $f(D)$ nach oben/unten beschränkt ist.

\subsection{Polynome und rationale Funktionen}

Eine Funktion $p : \R \to \R$ heißt \emph{(reelles) Polynom}, wenn es $n \in \N_0$ und $a_0,\ldots,a_n \in \R$ gibt, sodass
\[ p(x) = \sum_{k=0}^n a_k x^k \qquad \text{für alle } x \in \R \]
Der \emph{Grad} $n$ von $f$ erfüllt $a_i = 0~\Forall i > n$.

Ein Polynom $f$ vom Grad $n$ hat höchstens $n$ Nullstellen.
Beweisidee: Für jede Nullstelle $\alpha$ ist $(x - \alpha)$ ein Teiler von $f$. Es kann nicht mehr als $n$ solcher Teiler geben.

Haben wir mindestens $n+1$ verschiedene Stellen $x_0,\ldots,x_n$, dann ist das ein Polynom vom Grad $n$ dadurch eindeutig bestimmt.
Anders gesagt: Haben zwei Polynome vom Grad $\leq n$ mindestens $n+1$ gleiche Werte, so sind die Polynome gleich.

Zwei Wachstumssätze für ein Polynom $p$ mit Koeffizienten $a_i$:
\begin{itemize}
  \item Es gibt ein $r \in \R_+$, so dass
        \[ \frac{1}{2} \cdot a_n \cdot \abs{x}^n \leq \abs{p(x)} \leq 2 \cdot a_n \cdot \abs{x}^n \qquad\text{für alle } \abs{x} \geq r \]
  \item Sei $\rho > 0$. Dann gibt es ein $M > 0$ so dass
        \[ \abs{p(x)} \leq M \cdot \abs{x}^n \qquad\text{für alle } x \in \R, \abs{x} \geq \rho \]
\end{itemize}

\section{Funktionen: Grenzwerte und Stetigkeit}

\subsection{Grenzwerte}

Sei $D \subset \R$ nicht leer. Dann ist $x_0 \in \R$ ein \emph{Häufungspunkt von $D$}, wenn $D \cap (U_\rho(x_0) \setminus \Set{x_0})$ für jedes $\rho > 0$ nicht leer ist.
Ist $x_0 \in D$ aber kein Häufungspunkt von $D$, dann ist $x_0$ ein \emph{isolierter Punkt von $D$}.

Sei $f : D \to \R$ und $x_0$ Häufungspunkt von $D$. Dann ist $L \in \R$ der \emph{Grenzwert von $f$ für $x \to x_0$}, wenn
\[ \abs{f(x) - L} < \varepsilon \qquad \text{für alle } x \in D \text{ und } \abs{x-x_0} \in (0, \delta) \]
mit einem von $\varepsilon$ abhängigem $\delta > 0$. Man schreibt $\lim_{x \to x_0} f(x) = L$.

Grenzwerte von Funktionen kann man auch über Folgen beschreiben.
Sei $L$ der Grenzwert von $f$ für $x \to x_0$.
Äquivalent gilt: Für jede Folge $(x_n)_{n \geq 1}$ in $D \setminus \Set{x_0}$ mit $\lim_{n \to \infty} x_n = L$ gilt $\lim_{n \to \infty} f(x_n) = L$.

Damit erhalten wir auch für die Grenzwerte $\lim_{x \to x_0} f(x) = L_f$ und analog für $L_g$:
\begin{alignat*}{4}
  \lim_{x \to x_0} && (\alpha f)(x) &= \alpha L_f \\
  \lim_{x \to x_0} && (f + g)(x) &= L_f + L_g \\
  \lim_{x \to x_0} &&  (f \cdot g)(x) &= L_f \cdot L_g \\
  \lim_{x \to x_0} && (\frac{f}{g})(x) &= \frac{L_f}{L_g} &&\qquad \text{falls } L_g \neq 0
\end{alignat*}

Konvergenz kann es auch nur einseitig geben.
Sei $x_0$ Häufungspunkt von $D \cap (x_0, \infty)$ und $g = f{\big|}_{D \cap (x_0, \infty)}$.
Wenn $\lim_{x \to x_0} g(x) = L$, dann ist $f$ \emph{rechtsseitig konvergent gegen $L$ für $x \to x_0$}, geschrieben $\lim_{x \downarrow x_0} f(x) = L$.
Analog \emph{linksseitig konvergent} als  $\lim_{x \uparrow x_0} f(x) = L$ mit $D \cap (-\infty, x_0)$.

Divergenz geht wieder ähnlich wie bei den Folgen.
$f$ ist \emph{bestimmt divergent gegen $\infty$}, wenn es zu jedem $M > 0$ ein $R \in D$ gibt, so dass
\[ f(x) > M \qquad \text{für alle } x > R \]
Analog für $-\infty$.
Auch analog für rechtsseitige Grenzwerte mit $x \in D \cap (x_0,a)$ und $a \in D \cap (x_0, \infty)$, linksseitige Grenzwerte mit $x \in D \cap (b,x_0)$ und $b \in D \cap (-\infty,x_0)$.

\subsection{Stetige Funktionen}

Stetige Funktionen sind solche, die man ``ohne Sprünge in einer Linie'' zeichnen kann.

Sei $f : D \to \R$ und $x_0 \in D$.
\begin{itemize}
  \item Ist $x_0$ Häufungspunkt von $D$ und $\lim_{x \to x_0} f(x) = f(x_0)$, dann heißt $f$ \emph{stetig in $x_0$}.
  \item Ist $x_0$ kein Häufungspunkt von $D$, so ist $f$ auch stetig in $x_0$.
  \item $f$ ist stetig auf $E \subset D$, wenn $f$ auf jedem $x_0 \in E$ stetig ist.
\end{itemize}

Dass $f$ stetig in $x_0$ ist, ist äquivalent dazu, dass für jede Folge $(x_n)_{n \geq 1}$ in $D$ mit $\lim_{n \to \infty} x_n = x_0$ gilt:
\[ \lim_{n \to \infty} f(x_n) = f(x_0) \]

Seien $f : D_f \to \R$ und $g : D_g \to \R$ stetig in $x_0 \in D_f \cap D_g$. Dann sind auch stetig: $\alpha f$, $f + g$, $f \cdot g$ und $f / g$ (wenn $g(x_0) \neq 0$).

Sei $f : D_f \to \R$ stetig in $x_0$ und $g : D_g \to \R$ stetig in $f(x_0)$. Dann ist $g \circ f$ stetig in $x_0$.

Polynomfunktionen sind stetig auf $\R$.
Rationale Funktionen sind stetig auf ihrem maximalen Definitionsbereich.
$\exp$, $\sin$, $\cos$ sind stetig auf $\R$.

Wenn $f$ stetig in $x_0 \in D$ ist, dann gibt es eine Umgebung $U$ von $x_0$, so dass $f|_{D \cap U}$ beschränkt ist.

Wenn $f$ stetig und $f(x_0) > 0$ ($x_0 \in D$) ist, dann existiert eine Umgebung $U$ und ein $\rho > 0$, so dass $f(x) > \rho$  für alle $x \in D \cap U$.

Sei $f$ streng monton und stetig. Dann ist auch die Umkehrfunktion $f^{-1}$ stetig.

Sei $f : D \to \R$ stetig und $[a,b] \subset D$. Dann hat $f$ auf $[a,b]$ ein Minimum und ein Maximum.

\emph{Zwischenwertsatz.} Sei $f : D \to \R$ stetig und $[a,b] \subset D$ und $a < b$.
\begin{itemize}
  \item Gilt $f(a) < 0$ und $f(b) > 0$ (oder andersrum), dann existiert ein $c \in [a,b]$ mit $f(c) = 0$.
  \item Sei $m = \min \Set{ f(x) | x \in [a,b] }$ und $M = \max \Set{ f(x) | x \in [a,b] }$. Dann gilt $f([a,b]) = [m, M]$.
\end{itemize}

Mit dem Zwischenwertsatz kann man zeigen, dass jedes reelle Polynom ungeraden Grades eine Nullstelle in $\R$ hat.

Eine Verfeinerung der Stetigkeit ist die \emph{Lipschitz-Stetigkeit}.
Diese besteht, wenn es eine Konstante $L \in \R$ mit $L \geq 0$ gibt, sodass für alle $x,y \in D$ gilt:
\[ \abs{f(x) - f(y)} \leq L \abs{x-y} \]

\section{Differentialrechnung}

\subsection{Ableitungen}
\label{sec:ableitungen}

Zunächst der \emph{Differenzenquotient} für $f$, $x_0$ und $h \neq 0$:
\[ \frac{f(x_0 + h) - f(x_0)}{h} \]

Sei $f : D \to \R$ und $x_0$ innerer Punkt von $D$.
$f$ ist in $x_0$ \emph{differenzierbar}, wenn der folgende Grenzwert existiert:
\[ f'(x_0) := \lim_{h \to 0} \frac{f(x_0 + h) - f(x_0)}{h} \]
Ebenfalls gibt es einseitige Differenzierbarkeit mit $\lim_{h \uparrow 0}$ ($f_-(x_0)$) und $\lim_{h \downarrow 0}$ ($f_+(x_0)$).
$f$ ist \emph{differenzierbar auf $D$}, wenn $f$ in jedem $x_0 \in D$ differenzierbar ist.

Wenn $f$ bei $x_0$ differenzierbar ist, dann ist $f$ auch stetig in $x_0$.

$f'$, $f''$ und so weiter heißen die \emph{Ableitungen} einer entsprechend oft differenzierbaren Funktion $f$.

Rechenregeln für Differentiation. Mit $f, g$ differenzierbar auf $x_0$:
\begin{alignat*}{3}
  (\alpha f)'&(x_0) = \alpha f'(x_0)\qquad &&\Forall \alpha \in \R \\
  (f + g)'&(x_0) = f'(x_0) + g'(x_0) \\
  (f \cdot g)'&(x_0) = f'(x_0) \cdot g(x_0) + f(x_0) \cdot g'(x_0) \\
  \left(\frac{f}{g}\right)'&(x_0) = \frac{f'(x_0) g(x_0) - f(x_0) g'(x_0)}{g(x_0)^2}\qquad && \text{für } g(x_0) \neq 0 \\
  (g \circ f)'&(x_0) = g'(f(x_0)) \cdot f'(x_0)
\end{alignat*}
Diese Regeln kann man leicht zeigen, indem man die Definition der Ableitung einsetzt und dann die Grenzwertregeln verwendet.

Für eine stetige, injektive und differenzierbare Funktion $f$ gilt außerdem:
\[ (f^{-1})'(y_0) = \frac{1}{f'(f^{-1}(y_0))} \]

\subsection{Ableitungstabelle}

{\setlength\extrarowheight{5pt}
\begin{tabularx}{\textwidth}{M|M|M}
D & f : D \to \R & f' : D \to \R \\ \hline
\begin{array}{ll}
  \R, & \text{ falls } n \in \N_0 \\
  \R \setminus \Set{0}, & \text{ falls } n \in \mathbb{Z} \setminus \N_0 \\
  (0, \infty), & \text{ falls } n \in \R \setminus \mathbb{Z}
\end{array} & x^n & n \cdot x^{n-1} \\
\R & \exp(x) & \exp(x) \\
(0,\infty) & \ln(x) & \frac{1}{x} \\
\R & \sin(x) & \cos(x) \\
\R & \cos(x) & -\sin(x) \\
\R & \arctan(x) & \frac{1}{1+x^2} \\
\end{tabularx}}

\subsection{Der Mittelwertsatz der Differentialrechnung}

Wenn $f$ in $x_0$ ein relatives Extremum hat, dann gilt $f'(x_0) = 0$.

\emph{Satz von Rolle}. Sei $f : [a,b] \to \R$ stetig mit $a < b$ und $f$ auf $(a,b)$ differenzierbar mit $f(a) = f(b) = 0$. Dann existiert mindestens ein $x_0 \in (a,b)$ mit $f'(x_0) = 0$.

\emph{Mittelwertsatz}. Sei $f : [a,b] \to \R$ und $a < b$ und $f$ differenzierbar auf $(a,b)$. Dann existiert mindestens ein $x_0 \in (a,b)$ mit
\[ f'(x_0) = \frac{f(b) - f(a)}{b - a} \]
Anders formuliert: Die Steigung der Tangente in $x_0$ ist gleich der Steigung der Sekante durch $(a, f(a))$ und $(b, f(b))$.

Daraus folgen einige Sätze. Seien $I$ ein Intervall mit $\mathring{I} \neq \emptyset$ und $f : I \to \R$, $g : I \to \R$ beide stetig und auf $\mathring{I}$ differenzierbar.
\begin{alignat*}{3}
  \Forall x \in \mathring{I}: f'(x) = 0 &\Longleftrightarrow& ~\Exists c \in \R: (f(x) = c ~\Forall x \in I) \\
  \Forall x \in \mathring{I}: f'(x) = g'(x) &\Longleftrightarrow& ~\Exists c \in \R: (f(x) = g(x) + c ~\Forall x \in I) \\
  f \text{ ist monoton wachsend } &\Longleftrightarrow& f'(x) \geq 0 ~\Forall x \in \mathring{I} \\
  f \text{ ist monoton fallend } &\Longleftrightarrow& f'(x) \leq 0 ~\Forall x \in \mathring{I} \\
  f \text{ ist streng monoton wachsend } &\Longleftarrow& f'(x) > 0 ~\Forall x \in \mathring{I} \\
  f \text{ ist streng monoton fallend } &\Longleftarrow& f'(x) < 0 ~\Forall x \in \mathring{I}
\end{alignat*}

Endlich können wir auch Extrema berechnen.

Sei $\delta > 0$ so, dass $f$ in $U_\delta(x_0) \setminus \Set{ \delta }$ differenzierbar ist.
\medbreak
\begin{tabular}{ccc|c}
  $\Forall x \in U_\delta(x_0)$ & & $\Forall x \in U_\delta(x_0)$ & $\Longrightarrow$ \\
  mit $x > x_0$ & & mit $x < x_0$ &  \\
  \hline
  $f'(x) < 0$ &$\land$& $f'(x) > 0$ & $x_0$ ist strikte lokale Minimalstelle \\
  $f'(x) > 0$ &$\land$& $f'(x) < 0$ & $x_0$ ist strikte lokale Maximalstelle
\end{tabular}
\medbreak
Gilt entweder $f'(x) > 0$ oder $f'(x) < 0$ für alle $x \in U_\delta(x_0) \setminus \Set{x_0}$,\\ dann ist $x_0$ keine lokale Extremstelle.

Sei $f$ zweimal stetig differenzierbar in einer Umgebung von $x_0$ und $x_0$ innerer Punkt von $D$.
\begin{align*}
  f'(x_0) = 0 \land f''(x_0) > 0 &\Rightarrow x_0 \text{ ist strikte lokale Maximalstelle} \\
  f'(x_0) = 0 \land f''(x_0) < 0 &\Rightarrow x_0 \text{ ist strikte lokale Minimalstelle}
\end{align*}

\newpage
\section{Stammfunktionen und Integrale}

\subsection{Stammfunktionen}

Wir definieren Stammfunktionen als die Umkehrung der Ableitung.

$F : D \to \R$ ist eine \emph{Stammfunktion} von $f$, wenn $F$ auf $D$ differenzierbar ist mit $F' = f$.

Für zwei Stammfunktionen $F, G$ von $f$ existiert immer ein $c \in \R$, sodass $F = G + c$.

Für stetiges $f : I \to \R$ mit Stammfunktion $F$ schreiben wir auch $\Int{f(x)}{x,a,b} := F(b) - F(a) =: F|_a^b$.

Wir erhalten Rechenregeln für Integrale.
\begin{alignat*}{3}
  \Int{f(x)}{x,a,b} + \Int{f(x)}{x,b,c} &= \Int{f(x)}{x,a,c} \\
  \Int{\alpha f(x)}{x,a,b} &= \alpha \Int{f(x)}{x,a,b} \\
  \Int{f(x) + g(x)}{x,a,b} &= \Int{f(x)}{x,a,b} + \Int{g(x)}{x,a,b} \\
  \Int{f(t) \cdot g'(t)}{t,a,b} &= (f(t) \cdot g(t))|_a^b - \Int{f'(t) \cdot g(t)}{t,a,b} &&\qquad\textit{\small (partielle Integration)} \\
  \Int{f(\phi(t)) \cdot \phi'(t)}{t,a,b} &= \Int{f(x)}{x,\phi(a),\phi(b)} &&\qquad\textit{\small (Substitutionsregel)}
\end{alignat*}
Für die letzten zwei Regeln benötigen wir stetig differenzierbare $f,g,\phi$.
Partielle Integration entspricht der Produktregel der Differentiation und die Substitutionsregel der Kettenregel.

Das Schreibweise \emph{unbestimmtes Integral} $\Int{f(x)}{x} = F(x)$ bedeutet nicht Gleichheit, sondern nur, dass $\Int{f(x)}{x,a,b} = F(x)|_a^b$ gilt.

\subsection{Zerlegungssummen}

Nach dem Mittelwertsatz der Differentialrechnung gilt für stetiges $f$:
\[
  \Int{f(x)}{x,a,b} = F(a) - F(b) = f(\xi) \cdot (b - a) \qquad\text{für ein } \xi \in (a,b)
\]
Das Integral entspricht also dem Flächeninhalt zwischen Funktion und x-Achse ihres Graphen.

Über diese geometrische Interpretation können wir auch ein Integral direkt berechnen.
Dazu wird die Fläche unter der Funktion in viele kleine Rechtecke eingeteilt.

Die Breiten der Rechtecke sind durch eine \emph{Zerlegung} $Z$ eines Intervalls gegeben.
$[a,b]$ wird durch viele Punkte $x_i$ getrennt, wobei $x_0 = a$ und $x_n = b$.

Die Fläche der kleinen Rechtecke wird nun mit obiger Formel bestimmt, wobei für $\xi$ hier zwei Approximationen verfügbar sind.

{\setlength{\extrarowheight}{10pt}
\begin{tabularx}{\textwidth}{MM}
  m_i = \min_{x \in [x_{i-1}, x_i]} f(x) & M_i = \max_{x \in [x_{i-1}, x_i]} f(x) \\
  U(f,Z) = \sum_{i=1}^n m_i \cdot (x_i - x_{i-1}) & O(f,Z) = \sum_{i=1}^n M_i \cdot (x_i - x_{i-1}) \\
  U(f) = \sup_{Z \in \mathcal{Z}} U(f,Z) & U(f) = \inf_{Z \in \mathcal{Z}} O(f,Z) \\
  \text{\emph{Untersumme}} & \text{\emph{Obersumme}}
\end{tabularx}}
\medskip

Jetzt sind $U(f)$ und $O(f)$ also die besten Annäherungen an die jeweiligen Summen.
Genau wenn nun $U(f) = O(f)$ gilt, nennen wir $f$ \emph{integrierbar}:
\[ \Int{f(x)}{x,a,b} = U(f) = O(f) \]
Konvention schreibt für $a < b$ außerdem vor:
\[ \Int{f(x)}{x,b,a} = -\Int{f(x)}{x,a,b} \]

\subsection{Uneigentliche Integrale}

Es gibt zwei Arten von uneigentlichen Integralen: Solche mit unbeschränkten Integrationsintervallen und solche mit unbeschränkten Integranden.

Eine stetige Funktion $f$ ist \emph{uneigentlich integrierbar}, wenn $\lim_{b \to \infty} \Int{f(x)}{x,a,b}$ existiert. Wir schreiben $\Int{f(x)}{x,a,\infty} := \lim_{b \to \infty} \Int{f(x)}{x,a,b}$.
Analoge Definition für untere Schranke auf $-\infty$, sowie oben und unten.

Ist $f$ auf einer Grenze nicht definiert, aber existiert der Grenzwert, so können wir auch ein uneigentliches Integral aufstellen. Für $f : (a,b] \to \R$ und falls $\lim_{\alpha \downarrow a} \Int{f(x)}{x,\alpha,b}$ existiert, ist $f$ über $(a,b]$ \emph{uneigentlich integrierbar} und man schreibt auch einfach $\Int{f(x)}{x,a,b} := \lim_{\alpha \downarrow a} \Int{f(x)}{x,\alpha,b}$.
Analoge Definition für oben unbeschränkt.

Also aufpassen: Bei $\Int{f(x)}{x,a,b}$ prüfen, ob der Integrand an den Grenzen unbeschränkt ist.

\subsection{Gewöhnliche Differentialgleichungen}

\subsubsection{Separierbare Gewöhnliche Differentialgleichungen}

Seien $F : I \to \R$, $g : J \to \R$ stetig.

Eine \emph{separierbare gewöhnliche Differentialgleichung} hat die Form $y' = f(x) \cdot g(x)$.

$\phi$ ist eine \emph{Lösung} der Differentialgleichung, wenn gilt:
        \[ \phi'(x) = f(x) \cdot g(\phi(x)) \qquad \Forall x \in I' \subset I \]

Ein \emph{Anfangswertproblem} fordert zusätzlich $y_0 := y(x_0) = \phi(x_0)$ für ein $x_0 \in I'$.

\paragraph{Lösung über Umkehrfunktion}
Wenn $g(y_0) \neq 0$ gilt, kann man über zwei Integrationen und einmal die Bestimmung der Umkehrfunktion eine Lösung bestimmen.
\begin{itemize}
  \item $F : I \to \R$ Stammfunktion von $f$ mit $F(x_0) = 0$,
  \item $H : J' \to \R$ Stammfunktion von $\frac{1}{g}$ auf $J' \subset J$, \\
        mit $y_0 \in J'$ und $g(y) \neq 0~\Forall y \in J'$ und $H(y_0) = 0$.
\end{itemize}
Dann gibt es ein $I' \subset I$ mit eindeutiger Lösung: $\phi(x) = H^{-1}(F(x)) \quad \Forall~x \in I'$.

\subsubsection{Lineare Differentialgleichungen}

Seien $I \subset \R$, $x_0 \in I$ und $a, b : I \to \R$ stetig.
\bigskip

\begin{halfboxl}
\vspace{-\baselineskip}
\begin{tightcenter}\emph{Homogene Gleichung} \end{tightcenter}
\[ y' = a(x) \cdot y,\qquad y(x_0) = y_0 \]

\bigskip
\begin{tightcenter}\small{}Lösung: \end{tightcenter}
\[ \phi(x) = y_0 \cdot \exp(\Int{a(t)}{t,x_0,x}) \]
\end{halfboxl}%
\begin{halfboxr}
\vspace{-\baselineskip}
\begin{tightcenter}\emph{Inhomogene Gleichung} \end{tightcenter}
\[ y' = a(x) \cdot y + b(x),\qquad y(x_0) = y_0 \]

\bigskip
\begin{tightcenter}\small{}Lösung: \end{tightcenter}
\[ \psi(x) = \phi(x) \cdot u(x) \]
\[ \phi(x) = \exp(\Int{a(t)}{t,x_0,x}) \]
\[  u(x) = y_0 + \Int{\frac{b(t)}{\phi(t)}}{t,x_0,x} \]
\end{halfboxr}

\section{Funktionen mehrerer Veränderlicher}

Einigen Formalismus für Matrizen haben wir schon im \href{./la.pdf}{Panikzettel Lineare Algebra} notiert.

Für $2 \times 2$-Matrizen gilt:
\begin{itemize}
  \item $\det A := a_{11} \cdot a_{22} - a_{12} \cdot a_{21}$.
  \item Genau wenn $\det A \neq 0$ ist $A$ invertierbar und
        \[ A^{-1} = \frac{1}{\det A} \cdot \begin{pmatrix} a_{22} & -a_{12} \\ -a_{21} & a_{11} \end{pmatrix} \]
\end{itemize}

\subsection{Kurven}

Eine \emph{Kurve} ist ein Funktion $\phi : I \to \R^n$ mit $\phi_1, \ldots, \phi_n : I \to \R$ für die Komponenten.

Eine Kurve ist \emph{stetig/differenzierbar/stetig differenzierbar}, wenn alle $\phi_i$ diese Eigenschaft haben.

Die \emph{Ableitung} der Kurve $\phi$ ist
\[ \phi' : I \to \R^n, t \mapsto \begin{pmatrix} \phi_1'(t) \\ \vdots \\ \phi_n'(t) \end{pmatrix} \]

Das Bild $\phi(I) = \Set{ \phi(t) | t \in I}$ heißt \emph{Spur} von $\phi$.

\subsection{Längenmessung}

Als Verallgemeinerung des Absolutbetrags die \emph{Norm}:
\[ \norm{x} := \sqrt{x_1^2 + \ldots + x_n^2} \]

Für die Norm gilt wie für den Absolutbetrag:
\begin{itemize}
  \item $\norm{x} \geq 0$.
  \item $\norm{x} = 0 \iff x = 0$.
  \item $\norm{c \cdot x} = \abs{c} \cdot \norm{x}$.
  \item $\norm{x+y} \leq \norm{x} + \norm{y}$ \emph{(Dreiecksungleichung)}.
\end{itemize}

Die \emph{offene Kugel} für ein $a \in \R^n$ und $r > 0$:
\[ K_r(a) := \Set{x \in \R^n | \norm{x-a} < r } \]

Ein $U \subset \R^n$ heißt \emph{Umgebung von $a$}, wenn es ein $r > 0$ gibt, sodass $K_r(a) \subset U$.

$a \in M$ heißt \emph{innerer Punkt von $M$}, wenn es eine Umgebung $U$ von $a$ gibt, sodass $U \subset M$.
$M$ heißt \emph{offen}, wenn jeder Punkt von $M$ ein innerer Punkt ist.

\subsection{Stetigkeit}

Eine Kurve $f$ ist \emph{stetig in $a$} für ein $a \in M$, wenn es zu jedem $\varepsilon > 0$ ein $\delta > 0$ gibt, sodass $\norm{f(x) - f(a)} < \varepsilon$.

\subsection{Differentialrechnung}

\subsubsection{Partielle Differenzierbarkeit}

Die \emph{Richtungsableitung} ist
\[ D_v~f(a) := \lim_{t \to 0} \frac{1}{t}~(f(a+tv) - f(a)) \]
in Richtung $v \in \R^n \setminus \Set{0}$ von $f$ im Punkt $a$.

Die \emph{partielle Ableitung} einer Funktion in einem Punkt ist einfach die Richtungsableitung in Richtung $e_i$, wobei $e_i$ ein Einheitsvektor ist.
Wir schreiben auch $D_k~f(a) := \pderiv{f}{x_k} := D_{e_k}~f(a)$.

Die \emph{Funktionalmatrix/Jacobi-Matrix/Differential} von $F$ an der Stelle $x$, sowie der \emph{Gradient}:
\[
  D~f(x) := \begin{pmatrix} D_1~f(x) & \ldots & D_n~f(x) \end{pmatrix} \qquad\qquad \nabla f(a) := D~f(a)^T
\]
Damit können wir eine Richtungsableitung auch folgendermaßen berechnen:
\[ D_v~f(a) = D~f(a) \cdot v \]

Begriffe \emph{partiell differenzierbar} und \emph{stetig partiell differenzierbar} analog zu Begriffen in vorherigen Abschnitten.

\subsubsection{Totale Differenzierbarkeit}

$f$ heißt \emph{total differenzierbar} in $a \in U$ ($U$ offen), wenn man $f$ wie folgt darstellen kann:
\[ f(x) = f(a) + T \cdot (x - a)  + \norm{x-a} \cdot \phi(x) \]
Mit einer Matrix $T \in \R^{m \times n}$ und einer stetigen Funktion $\phi : U \to \R^m$.

Äquivalent: Es gibt ein $T \in \R^{m \times n}$, sodass
\[ \lim_{x \to a} \frac{1}{\norm{x-a}} (f(x) - f(a) - T \cdot (x-a)) = 0 \]

Ist $f$ total differenzierbar in $a$, dann ist $f$ auch partiell differenzierbar in $a$. Außerdem ist $T$ eindeutig bestimmt durch:
\[
  T_{i,j} = ((D~f)(a))_{i,j} = \pderiv{f_i}{x_j}(a)
\]

$f$ ist \emph{total differenzierbar}, wenn $f$ in allen Punkten $a \in U$ total differenzierbar ist.

Die Kettenregel:
\begin{alignat*}{3}
  && (D~(g \circ f))(a) &= (D~g)(f(a)) \cdot (D~f)(a) \\
\Leftrightarrow\quad && \pderiv{(g \circ f)_i}{x_j}(a) &= \sum_{k=1}^m \pderiv{g_i}{y_k}(f(a)) \cdot \pderiv{f_k}{x_j}(a)
\end{alignat*}

Wenn $f : U \to \R^n$ injektiv und stetig diffbar ist, $f(U)$ offen und $f^{-1}$ diffbar, dann gilt:
\[ D~f^{-1}(y) = (D~f(f^{-1}(y)))^{-1} \]

\subsubsection{Der Satz von der Umkehrfunktion}

Sei $f : M \to \R^n$ stetig diffbar und $a \in M$ so, dass $D~f(a)$ invertierbar ist. \\
Dann existieren offene $U \subset M$ mit $a \in U$ und $V \subset f(M)$ mit $f(a) \in V$ so, dass
\begin{itemize}
  \item $f|_U : U \to V$ bijektiv ist,
  \item Für alle $x \in U$ ist $D~f(x)$ invertierbar,
  \item $(f|_U)^{-1}$ ist stetig diffbar,
  \item $(D~f^{-1})(y) = (D~f)(f^{-1}(y))^{-1}$.
\end{itemize}

\subsubsection{Der Satz über implizite Funktionen}

Dieser Satz erlaubt das Umstellen von komplizierteren Gleichungssystemen nach bestimmten Parametern, sodass das System als Nullstellenproblem in einer Umgebung der Lösung gelöst werden kann:
\[ f \begin{pmatrix} x \\ y \end{pmatrix} = 0 \quad\Leftrightarrow\quad f \begin{pmatrix} x \\ g(x) \end{pmatrix} = 0 \]
Hier wurde nach $x$ aufgelöst, sodass $y$ als $g(x)$ dargestellt wird.

Sei unsere Umgebung nun um $c$. Wir setzen $D_x~f$ als die Jacobi-Matrix mit Spalten für Parameter, nach denen aufgelöst wird, sowie $D_y~f$ analog für implizit dargestellte Parameter.

Voraussetzungen:
\begin{itemize}
  \item $f : M \to R^{m+n}$ ist stetig diffbar,
  \item $f(c) = 0$,
  \item $(D_y~f)(c)$ ist invertierbar (bspw. über $\det (D_y~f)(c) \neq 0$ zeigen).
\end{itemize}

Dann sagt der \emph{Satz über implizite Funktionen}: Es gibt ein $g : U \to V$ mit $U$ Umgebung von $a$, $V$ Umgebung von $f(a)$, gegeben durch die Ableitung:
\[ (D~g)(x) = - \left( (D_y~f) \begin{pmatrix} x \\ g(x) \end{pmatrix} \right)^{-1} \cdot \left( (D_x~f) \begin{pmatrix} x \\ g(x) \end{pmatrix} \right) \]

\subsection{Gewöhnliche Differentialgleichungen: Existenz und Eindeutigkeit}

Ein \emph{System von $n$ gewöhnlichen Differentialgleichungen} mit $D \subset \R \times \R^n$ offen und $f : D \to \R^n$:
\begin{tightcenter} $y' = f(x,y)$. \end{tightcenter}

Eine Lösung ist eine Kurve $\phi : I \to \R^n$ mit:
\begin{itemize}
  \item $\phi$ ist diffbar,
  \item $(x, \phi(x)) \in D \quad\Forall x \in I$,
  \item $f(x, \phi(x)) \in D \quad\Forall x \in I$.
\end{itemize}

Wir haben nur einen Satz zur Existenz von Lösungen: \\
Wenn $f$ stetig ist und alle partiellen Ableitungen $\pderiv{f_i}{x_j}$ existieren und stetig sind, dann hat jedes Anfangswertproblem eine eindeutige Lösung auf einem passenden Intervall.

\end{document}
