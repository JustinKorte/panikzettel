\documentclass[english]{panikzettel}

\title{Probabilistic Programming Panikzettel}
\author{Philipp Schröer, Caspar Zecha}

\usepackage[nameinlink,noabbrev]{cleveref}

\usepackage{bussproofs}
\usepackage{listings}
\lstset{mathescape=true}

\usepackage{array}
\newcolumntype{L}{>{$}l<{$}}
\newcolumntype{C}{>{$}c<{$}}
\newcolumntype{R}{>{$}r<{$}}

\newcommand{\Dist}{\mathrm{Dist}}
\newcommand{\Paths}{\mathrm{Paths}}
\newcommand{\Cyl}{\mathrm{Cyl}}
\renewcommand{\Pr}{\mathrm{Pr}}
\newcommand{\Pre}{\mathrm{Pre}}
\newcommand{\ER}{\mathrm{ER}}

\DeclarePairedDelimiter\sem{\llbracket}{\rrbracket}
\DeclarePairedDelimiter\ssem{[}{]}
\DeclarePairedDelimiter{\angled}{\langle}{\rangle}

\newcommand{\stmtSkip}{\texttt{skip}}
\newcommand{\stmtDiverge}{\texttt{diverge}}
\newcommand{\stmtAsgn}[2]{#1 := #2}
\newcommand{\stmtObserve}[1]{\texttt{observe (}#1\texttt{)}}
\newcommand{\stmtRasgn}[2]{#1 :\approx #2}
\newcommand{\stmtSeq}[2]{#1;~ #2}
\newcommand{\stmtIf}[3]{\texttt{if}~(#1)~\{ #2 \}~\texttt{else}~\{ #3 \}}
\newcommand{\stmtNondet}[2]{#1~[]~#2}
\newcommand{\stmtProb}[3]{#2 ~[#1]~ #3}
\newcommand{\stmtWhile}[2]{\texttt{while}~(#1)~\{ #2 \}}

\newcommand{\fix}{\mathrm{fix}~}
\newcommand{\lfp}{\mathrm{lfp}~}
\newcommand{\gfp}{\mathrm{gfp}~}

\newcommand{\Vars}{\mathrm{Vars}}
\newcommand{\lam}[1]{\lambda #1.~}

\renewcommand{\wp}{\mathrm{wp}}
\newcommand{\wlp}{\mathrm{wlp}}
\newcommand{\cwp}{\mathrm{cwp}}
\newcommand{\ert}{\mathrm{ert}}

\newcommand{\rat}{\mathbb{Q}}
\newcommand{\rel}{\mathbb{R}}
\newcommand{\relg}{\rel_{\geq 0}}

\newcommand{\down}{\downarrow}

\begin{document}

\maketitle

\setcounter{tocdepth}{2}
\tableofcontents

\newpage
\section{Introduction}

This Panikzettel is about the lecture Probabilistic Programming by Prof.\ Katoen held in the winter semester 2018/2019.

This Panikzettel is Open Source. We appreciate comments and suggestions at \\ \url{https://git.rwth-aachen.de/philipp.schroer/panikzettel}.

\section{Markov Chains}

\begin{halfboxl}
    \emph{Markov chains} are essential in our definition of the semantics of the pGCL probabilistic programming language.
    A Markov chain is a transition system with a state set, an initial state and a transition probability function between states.

    So instead of a simple transition relation between states as in usual transition systems, transitions now additionally have a probability.
    The probability of a transition from $\sigma_1$ to $\sigma_2$ is given by:
    \[
        \underbrace{\mathbf{P}(\sigma_1)}_{\Dist(\Sigma)}(\sigma_2) \subseteq [0,1]
    \]

    If $\Sigma$ is finite, we can also write $\mathbf{P}$ as a \emph{transition probability matrix}.
    This matrix is a square, stochastic matrix, i.e.\ each row sums to one.

\end{halfboxl}%
\begin{halfboxr}
    \vspace{-\baselineskip}
    \begin{defi}{Probability distribution}
        A \emph{probability distribution} on a countable set $X$ is a function $\mu : X \to [0,1] \subseteq \rel$ such that $\sum_{x \in X} \mu(x) = 1$.

        We call $\Set{ x | \mu(x) > 0 }$ the \emph{support set} of $\mu$.
        Let $\Dist(X)$ denote the set of probability distributions on $X$.
    \end{defi}

    \begin{defi}{Markov chain}
        A \emph{Markov chain} (MC) $D$ is a triple $(\Sigma, \sigma_l, \mathbf{P})$:
        \begin{itemize}[leftmargin=*]
            \item $\Sigma$ being a countable set of \emph{states},
            \item $\sigma_I \in \Sigma$ the \emph{initial state},
            \item $\mathbf{P} : \Sigma \to \Dist(\Sigma)$ the \emph{transition probability function}.
        \end{itemize}
    \end{defi}
\end{halfboxr}

\begin{halfboxl}
    A program execution will be a \emph{path} through the Markov chain.
    A path in a Markov chain is defined as a (possibly infinite) sequence of states where each single transition must have a probability larger than zero.

    We define the \emph{cylinder set} of a finite path $\hat{\pi}$ as all infinite paths with prefix $\hat{\pi}$.

    We can now define a \emph{probability distribution} on cylinder sets $\Pr$.
    Given a path finite $\hat{\pi}$, $\Pr(\Cyl(\hat{\pi}))$ is defined as the probability of the transitions between the finite prefix path states.

    Note that we define $\Pr$ only on cylinder sets, i.e.\ sets of infinite paths.
    However, the product is finite and only requires probabilities of the finite path.
\end{halfboxl}%
\begin{halfboxr}
    \vspace{-\baselineskip}
    \begin{defi}{Paths}
        $\pi = \sigma_0 \sigma_1 \ldots$ is a \emph{path} through MC $D$ where
        \begin{tightcenter}$\mathbf{P}(\sigma_i, \sigma_{i+1}) > 0 \quad \Forall i \in \nat$\end{tightcenter}

        Let $\Paths(D)$ denote the set of paths in $D$ starting in $\sigma_I$.
    \end{defi}

    \begin{defi}{Cylinder set}
        Let $\hat{\pi} = \sigma_0 \sigma_1 \ldots \sigma_n$ be a finite path in MC $D$.
        Then we define the \emph{cylinder set} $\Cyl(\hat{\pi})$:
        \footnotesize{}
        \[ \Cyl(\hat{\pi}) = \Set{ \pi \in \Paths(D) | \hat{\pi} \text{ is a prefix of } \pi } \]
    \end{defi}

    \begin{defi}{Cylinder probability}
        $\Pr$ is the unique \emph{probability distribution on cylinder sets}.
        ($\mathbf{P}(\sigma_0) = 1$ iff $\sigma_0 = \sigma_I$).
        \[
            \Pr(\Cyl(\sigma_0 \ldots \sigma_n)) = \prod_{0 \leq i < n} \mathbf{P}(\sigma_i, \sigma_{i+1})
        \]
    \end{defi}
\end{halfboxr}

\subsection{Reachability}

\begin{halfboxl}
    With the definition of the probability of a path, we now consider the more general problem of \emph{reachability}: What is the probability to reach a set of states $G \subseteq \Sigma$ in MC $D$?

    In our definition of reachability, this asks for the probability of all infinite paths containing a state in $G$ anywhere.
\end{halfboxl}%
\begin{halfboxr}
    \vspace{-\baselineskip}
    \begin{defi}{Reachability}
        Let MC $D$ with countable state space $\Sigma$ and $G \subseteq \Sigma$ the set of \emph{goal states}.
        The event \emph{eventually reaching $G$} is defined by:
        \small{}
        \[ \diamond G = \Set{ \pi \in \Paths(D) | \Exists i \in \nat.~ \pi[i] \in G } \]
    \end{defi}
\end{halfboxr}

If the Markov chain $D$ has a finite state space, we can calculate the reachability probability by solving a linear equation system.
We write $D_\sigma$ to mean the MC $D$ with initial state $\sigma$ and
\smallskip
\begin{tightcenter}$
    \Pr(\sigma \models \diamond G) = \Pr_\sigma(\diamond G) = \Pr(\Set{\pi \in \Paths(D_\sigma) | \pi \in \diamond G }).
$\end{tightcenter}
\medskip

\begin{theo}{Finite reachability solution}
    To solve $\Pr(\sigma \models \diamond G)$, define
    \begin{itemize}
        \item $\Sigma_? = \Pre^\ast(G) \setminus G$, the set of states that can reach $G$ in $> 0$ steps,
        \item $A = ( \mathbf{P}(\sigma, \tau) )_{\sigma,~\tau \in \Sigma_?}$, the transition probabilities in $\Sigma_?$, and
        \item $b = (b_\sigma)_{\sigma \in \Sigma_?}$, the probabilities to reach $G$ in exactly one step, i.e.\ $b_\sigma = \sum_{\gamma \in G} \mathbf{P}(\sigma, \gamma)$.
    \end{itemize}

    Then $x = (x_\gamma)_{\gamma \in \Sigma_?}$ with $x_\sigma = \Pr(\sigma \models \diamond G)$ is the unique solution of
    \smallskip
    \begin{tightcenter}$
        x = A \cdot x + b \qquad \text{\footnotesize{}or, equivalently} \qquad (I - A) \cdot x = b.
    $\end{tightcenter}
\end{theo}

\subsection{State Classification}

\begin{halfboxl}
    We will now classify states based on recurrence: Whether the MC is almost sure to return to a state (\emph{recurrent}) or not (\emph{transient}).

    Note that the \emph{first visit probability} requires a \emph{first visit}, so only paths that contain $\tau$ exactly once at their end are considered.

    The \emph{return probability} also requires a first return.
    It can be calculated by summing up all first visit probabilities.

    We now call a state $\sigma$ \emph{recurrent} if $f_\sigma = 1$ and \emph{transient} if $f_\sigma < 1$.
    For a transient state $\sigma$, we say the MC \emph{almost surely} returns to $\sigma$.
\end{halfboxl}%
\begin{halfboxr}
    \vspace{-\baselineskip}
    \begin{defi}{First visit probability}
        Let states $\sigma, \tau \in \Sigma$.

        We define $f^{(n)}_{\sigma,\tau}$ as the probability of a first visit to $\tau$ after exactly $n$ steps from $\sigma$.
    \end{defi}

    \begin{defi}{Return probability}
        Let states $\sigma, \tau \in \Sigma$.

        The \emph{return probability} $f_\sigma^{(n)}$ is defined as the probability of the first return to $\sigma$ (from $\sigma$) after exactly $n$ steps.
        This is equivalent to:
        \begin{tightcenter}\large{}$f_\sigma = \sum_{i=1}^\infty f_{\sigma,\sigma}^{(n)}$.\end{tightcenter}
    \end{defi}
\end{halfboxr}

\begin{halfboxl}
    For a recurrent state $\sigma$, we also define a \emph{mean recurrence time}: The expected number of steps between two successive visits to $\sigma$.

    This gives rise to the terms \emph{null} and \emph{postive recurrent}.
    It takes infinite on average to return to a null recurrent state, while positive recurrent states have a finite mean recurrence time.
\end{halfboxl}%
\begin{halfboxr}
    \vspace{-\baselineskip}
    \begin{defi}{Mean recurrence time}
        The \emph{mean recurrence time} $m_\sigma$ of a recurrent state $\sigma$ is
        \begin{tightcenter}$
            m_\sigma = \sum_{n=1}^\infty n \cdot f_\sigma^{(n)}
        $\end{tightcenter}

        If $m_\sigma < \infty$, we call $m_\sigma$ \emph{positive recurrent}, otherwise \emph{null recurrent}.
    \end{defi}
\end{halfboxr}

\begin{halfboxl}
    If the MC is finite, then we have a few properties w.r.t.\ recurrence:
    \begin{enumerate}
        \item Every state in a finite MC is either positive recurrent or transient.
        \item At least one state in a finite MC is positive recurrent.
        \item A finite MC has no null recurrent states.
    \end{enumerate}

    To show these properties, we use Foster's theorem.
\end{halfboxl}%
\begin{halfboxr}
    \vspace{-\baselineskip}
    \begin{theo}{Foster's theorem}
        A countable Markov chain is \emph{non-dissipative} if almost every infinte path eventually enters and remains in positive recurrent states.

        If the following conditions hold, the MC is non-dissipative:
        \begin{tightcenter}$
            \sum_{j \geq 0} j \cdot \mathbf{P}(i,j) \leq i \qquad \Forall \text{states } i
        $\end{tightcenter}
    \end{theo}
\end{halfboxr}

\begin{halfboxl}
    We can also classify Markov chains by periodicity.
    The definition is a bit tricky, so read it carefully.

    Further a state is \emph{ergodic} if it is positive and aperiodic.
    An MC is ergodic if all its states are ergodic.
\end{halfboxl}%
\begin{halfboxr}
    \vspace{-\baselineskip}
    \begin{defi}{Periodic state}
        A state $\sigma$ is \emph{periodic} if
        \begin{tightcenter}\small{}$
            f_\sigma^{(n)} > 0 \text{ implies } n = k \cdot d \text{ where period } d > 1.
        $\end{tightcenter}
        A state is \emph{aperiodic} otherwise.
    \end{defi}
\end{halfboxr}

At this point we may notice that mutually reachable states must have the same types.
More formally, if $\sigma$ and $\tau$ are two mutually reachable states, then being transient, null-recurrent, positive recurrent and $d$-periodic holds for $\tau$ if the respective property holds for $\sigma$.

If a MC is \emph{irreducible}, that is all states are mutually reachable, we can use Markov's theorem.

\begin{theo}{Markov's theorem}
    A finite, irreducible MC $D$ is positive recurrent.

    If $D$ is also aperiodic, then $D$ is ergodic and $\mathbf{P}^{\infty} = \lim_{n \to \infty} \mathbf{P}^n = \begin{pmatrix} v \\ \vdots \\ v \end{pmatrix}$ where $v = \begin{pmatrix} \frac{1}{m_1}, \ldots, \frac{1}{m_k} \end{pmatrix}$ and $k = \abs{\Sigma}$.
\end{theo}

The \emph{stationary distribution} of MC $D$ is a probability vector $x$ where $x = x \cdot \mathbf{P}$.
\[
    x_\sigma = \sum_{\tau \in \Sigma} x_\tau \cdot \mathbf{P}(\tau, \sigma) \quad \text{iff} \quad \underbrace{x_\sigma \cdot (1 - \mathbf{P}(\sigma, \sigma))}_\text{outflow of $\sigma$} = \underbrace{\sum_{\tau \neq \sigma} x_\tau \cdot \mathbf{P}(\tau, \sigma)}_\text{inflow of $\gamma$}
\]
An irreducible, positive recurrent MC has a unique stationary distribution satisfying $x_\sigma = \frac{1}{m_\sigma}$ for every state $\sigma$. If $\mathbf{P}$ is ergodic, then each row of $\mathbf{P}^\infty$ equals the limiting (stationary) distribution.

\subsection{Rewards}

\begin{halfboxl}
    We can also attach \emph{rewards} to states of Markov chains.
    The reward $r(\sigma)$ is the reward earned on leaving the state $\sigma$.
    We can also calculate a \emph{cumulative reward for reachability}.
\end{halfboxl}%
\begin{halfboxr}
    \vspace{-\baselineskip}
    \begin{defi}{MC with rewards}
        A \emph{reward MC} is a pair $(M,r)$ with $D$ an MC with state space $\Sigma$ and a \emph{reward function} $r : \Sigma \to \rel$.
    \end{defi}
\end{halfboxr}

\begin{defi}{Cumulative reward for reachability}
    Let $\pi = \sigma_0 \ldots \sigma_n$ be a finite path in $(D, r)$ and $G \subseteq \Sigma$ a set of \emph{target states} with $\pi \in \diamond G$.
    The \emph{cumulative reward} along $\pi$ until reaching $G$ is:
    \begin{tightcenter}$
        r_G(\pi) = r(\sigma_0) + \ldots + r(\sigma_{k-1}) \quad \text{where } \sigma_i \notin G \text{ for all } i < k \text{ and } \sigma_k \in G.
    $\end{tightcenter}

    If $\pi \notin \diamond G$, then $r_G(\pi) = 0$.
\end{defi}

\begin{defi}{Expected reward for reachability}
    The \emph{expected reward} for reachability until reaching $G \subseteq \Sigma$ from $\sigma \in \Sigma$ is:
    \[
        \ER(\sigma, \diamond G) = \sum_{\pi \models \diamond G} \Pr(\hat{\pi}) \cdot r_G(\hat{\pi})
    \]
    where $\hat{\pi} = \sigma_0 \ldots \sigma_k$ is the shortest prefix of $\pi$ such that $\sigma_k \in G$ and $\sigma_0 = \sigma$.
\end{defi}

\begin{defi}{Conditional expected reward}
    Let $\ER(\sigma, \diamond G | \neg \diamond F)=\frac{\ER(\sigma, \diamond G \cap \neg \diamond F)}{\Pr(\neg \diamond F)}$ be the \emph{conditional expected reward} until reaching $G$ under the condition that no states in $F \subseteq \Sigma$ are visited.
\end{defi}

\section{Probabilistic GCL (pGCL)}

\begin{halfboxl}
    Elementary ingredients for pGCL are
    \begin{itemize}
        \item Program variables $x \in \Vars$ whose values are fractional numbers,
        \item Arithmetic expressions $E$ over the program variables,
        \item Boolean expressions $G$ (guarding choice or loop) over the program variables,
        \item \emph{Distribution expressions} $\mu : \Sigma \to \Dist(\rat)$,
        \item \emph{Probability expressions} $p : \Sigma \to [0,1] \cap \rat$.
    \end{itemize}
\end{halfboxl}%
\begin{halfboxr}
    \vspace{-\baselineskip}
    \begin{defi}{pGCL syntax}
        \footnotesize{}
        \vspace{-\baselineskip}
        \begin{align*}
            \stmtSkip &&\text{empty statement}\\
            \stmtDiverge &&\text{divergence} \\
            \stmtAsgn{x}{E} &&\text{assignment} \\
            \stmtRasgn{x}{\mu} &&\text{random assignment}\\
            \stmtSeq{P1}{P2} &&\text{sequential composition} \\
            \stmtIf{G}{P1}{P2}&&\text{choice} \\
            \stmtProb{p}{P1}{P2}&&\text{probabilistic choice}\\
            \stmtWhile{G}{P}&&\text{iteration}
        \end{align*}
    \end{defi}
\end{halfboxr}

For random assignment $\stmtRasgn{x}{\mu}$ we evaluate the distribution expression $\mu$ in the curent program state $s$.
Then we sample from the resulting distribution $\mu(s)$ yielding value $v$ with probability $\mu(s)(v)$ and assign the value $v$ to $x$.

To prove correctness of programs we use formal semantics.
There are different kind of semantics.
We use operational semantics of pGCL on Markov chains to model the execution behaviour of a program.

\begin{defi}{pGCL operational semantics} \label{def:pgcl-semantics}
    The behaviour of a pGCL program $P$ is modelled by the MC $\sem{P}$:
    \begin{itemize}
        \item States are of the form
            \begin{itemize}
                \item $\angled{Q,s}$ where $Q$ is the remaining program to be executed,
                \item or $s = \lightning$ for violation of an observation (\stmtObserve{G}),
                \item or $\angled{sink}$ for successful program termination.
            \end{itemize}
        \item $s : \Vars \to \rat$ is a \emph{variable valuation}.
        \item $\sigma_I = \angled{P,s}$ is the initial state where $s$ fulfills the initial conditions.
        \item The transition relation $\rightarrow$ is the smallest relation satisfying the SOS rules below.
    \end{itemize}
    The \emph{output} of the program $P$ is the unique probability distribution given by $\lam{s} \Pr(s \models \diamond \angled{\downarrow, \cdot})$.


    \[
        \angled{\down,s} \rightarrow \angled{sink}
        \qquad
        \angled{sink} \rightarrow \angled{sink}
    \]
    \[
        \angled{\stmtSkip,s} \rightarrow \angled{\down,s}
        \qquad
        \angled{\stmtDiverge,s} \rightarrow \angled{\stmtDiverge,s}
    \]
    \[
        \angled{\stmtAsgn{x}{E},s} \rightarrow \angled{\down,s[x:=s(\sem{E})]}
    \]
    \begin{prooftree}
        \AxiomC{$\mu(s)(v)=a>0$}
        \UnaryInfC{$\angled{\stmtRasgn{x}{\mu},s} \xrightarrow{a} \angled{\down,s[x:=v]}$}
    \end{prooftree}


    \[
        \angled{\stmtProb{p}{P}{Q},s} \xrightarrow{p} \angled{P,s}
        \qquad
        \angled{\stmtProb{p}{P}{Q},s} \xrightarrow{1-p} \angled{Q,s}
    \]

    \begin{prooftree}
        \AxiomC{$\angled{P,s} \xrightarrow{a} \angled{P',s'}$}
        \UnaryInfC{$\angled{\stmtSeq{P}{Q}, s} \xrightarrow{a} \angled{\stmtSeq{P'}{Q},s'}$}
        \DisplayProof
        \hskip 1.5em
        \AxiomC{$\angled{Q,s} \xrightarrow{a} \angled{Q',s'}$}
        \UnaryInfC{$\angled{\stmtSeq{\downarrow}{Q}, s} \xrightarrow{a} \angled{Q',s'}$}
    \end{prooftree}


    \begin{prooftree}
        \AxiomC{$s \models G$}
        \UnaryInfC{$\angled{\stmtIf{G}{P}{Q}, s} \rightarrow \angled{P,s}$}
        \DisplayProof
        \hskip 1.5em
        \AxiomC{$s \not\models G$}
        \UnaryInfC{$\angled{\stmtIf{G}{P}{Q}, s} \rightarrow \angled{Q,s}$}
    \end{prooftree}

    \begin{prooftree}
        \AxiomC{$s \models G$}
        \UnaryInfC{$\angled{\stmtWhile{G}{P}, s} \rightarrow \angled{\stmtSeq{P}{\stmtWhile{G}{P}},s}$}
        \DisplayProof
        \hskip 1.5em
        \AxiomC{$s \not\models G$}
        \UnaryInfC{$\angled{\stmtWhile{G}{P}, s} \rightarrow \angled{\downarrow,s}$}
    \end{prooftree}

    \bigskip
    \begin{center}
        \footnotesize\sffamily
        \dotfill \hyperref[sec:cpGCL]{\textbf{cpGCL} (section~\ref{sec:cpGCL})}\dotfill
    \end{center}

    \begin{prooftree}
        \AxiomC{$s \models G$}
        \UnaryInfC{$\angled{\stmtObserve{G},s} \rightarrow \angled{\down,s}$}
        \DisplayProof
        \hskip 1.5em
        \AxiomC{$s \not\models G$}
        \UnaryInfC{$\angled{\stmtObserve{G},s} \rightarrow \angled{\lightning}$}
    \end{prooftree}

    \begin{prooftree}
        \AxiomC{$\angled{\lightning} \rightarrow \angled{sink}$}
        \DisplayProof
        \hskip 1.5em
        \AxiomC{$\angled{P,s} \rightarrow \angled{\lightning}$}
        \UnaryInfC{$\angled{\stmtSeq{P}{Q},s} \rightarrow \angled{\lightning}$}
    \end{prooftree}
\end{defi}

pGCL as defined above does not feature recursion, but we define two additional statements to define functions, which we call \emph{processes}.
We write $P = P_1$ to define a process $P$ that executes program $P_1$ and write $\texttt{call } P$ to call $P$.
Introducing recursion does not increase expressive power.

\iffalse
\begin{defi}{Pushdown Markov Chain}
    A pushdwn Markov chain $D$ is a tuple $(\Sigma, \sigma_i,\Gamma, \gamma_0, \Delta)$ where:
    \begin{itemize}
        \item $\Sigma$ is a countable set of (control) states
        \item $\sigma_i \in \sigma$ is the initial (control) state
        \item $\Gamma$ is a finite stack alphabet
        \item $\gamma_0 \in \Gamma$ is the bottom-of-the-stack symbol
        \item $\Delta: \Sigma \times \Gamma \to \Dist{\Sigma} \times (\Gamma \setminus \{\gamma_0\})$
    \end{itemize}
\end{defi}
\fi

\section{Domain Theory}

\begin{halfboxl}
    \vspace{-\baselineskip}
    \begin{defi}{Partial order}
        A \emph{partial order} $(D, \sqsubseteq)$ has a domain $D$ and a relation ${\sqsubseteq} \subseteq D \times D$, where {\small{}$\Forall d_1, d_2, d_3 \in D$}:

        \small{}
        \begin{tightcenter}
            {\footnotesize{}\textsc{Reflexivity}} \\
            $d_1 \sqsubseteq d_1$

            {\footnotesize{}\textsc{Transitivty}} \\
            $d_1 \sqsubseteq d_2 \land d_2 \sqsubseteq d_3 ~\Rightarrow~ d_1 \sqsubseteq s_3$

            {\footnotesize{}\textsc{Antisymmetry}} \\
            $d_1 \sqsubseteq d_2 \land d_2 \sqsubseteq d_1 ~\Rightarrow~ d_1 = d_2$
        \end{tightcenter}
    \end{defi}

    In pGCL semantics, loops are defined as fixed points of functions.
    We will need domain theory to prove existence of these fixed points and to approximate them.

    A \emph{complete lattice} is a partial order with upper bounds for all subsets (also called \emph{supremum}).
    Equivalently, one can require lower bounds for all subsets (also called \emph{infimum}).
    \bigskip

    A \emph{chain} $S \subseteq D$ comprises only ordered elements: {\footnotesize{}  $\Forall d_1, d_2 \in S$:\quad $d_1 \sqsubseteq d_2$ or $d_2 \sqsubseteq d_1$}.

    If $F : D \to D'$ is a monotonic function between complete lattices and $S \subseteq D$ is a chain in $D$, then
    \begin{tightcenter}
        $F(S) := \Set{ F(d) | d \in S }$ is a chain in $D'$
    \end{tightcenter}
    and \quad $\bigsqcup F(S) \sqsubseteq_{D'} F(\bigsqcup S)$.
\end{halfboxl}%
\begin{halfboxr}
    \vspace{-\baselineskip}
    \begin{defi}{(Least) Upper bound,\\\hphantom{Definition: }(Greatest) Lower bound}
        Let $(D, \sqsubseteq)$ be a partial order with $S \subseteq D$.
        \vspace{0.5\baselineskip}

        \begin{enumerate}[leftmargin=*]
            \item $d \in D$ is an \emph{upper bound} of $S$ ($S \sqsubseteq d$) if \\ $s \sqsubseteq d~\Forall s \in S$.
            \item $d$ is a \emph{least upper bound of $S$} ($d = \bigsqcup S$) if\\ $d \sqsubseteq d'$ for every upper bound $d'$ of $S$.
        \end{enumerate}
        \vspace{0.5\baselineskip}

        Analogous definitions for \emph{lower bound} and \emph{greatest lower bound}.
    \end{defi}

    \begin{defi}{Complete lattice}
        A \emph{complete lattice} is a
        \begin{itemize}
            \item partial order $(D, \sqsubseteq)$,
            \item such that all $S \subseteq D$ have \\
                  least upper bounds, {\small{}or equivalently,}\\ greatest lower bounds.
        \end{itemize}

        The least element is $\bot := \bigsqcup \emptyset$. \\
        The greatest element is $\top := \bigsqcap \emptyset$.
    \end{defi}

    \begin{defi}{Monotonicity}
        Let $(D, \sqsubseteq)$ and $(D', \sqsubseteq')$ be partial orders.
        $\Phi : D \to D'$ is \emph{monotonic} if {\small{}$\Forall d_1, d_2 \in D$}:
        \begin{tightcenter}
            $d_1 \sqsubseteq d_2 \Rightarrow \Phi(d_1) \sqsubseteq' \Phi(d_2)$.
        \end{tightcenter}
    \end{defi}
\end{halfboxr}

\begin{halfboxl}
    A \emph{(Scott-)continuous} function is a generalisation of the continuity we know from analysis to complete lattices.
    Every continuous function is monotonic.

    \begin{defi}{Fixed point}
        $d$ is a \emph{fixed point} of $\Phi : D \to D$ if $\Phi(d) = d$.
    \end{defi}
\end{halfboxl}%
\begin{halfboxr}
    \vspace{-\baselineskip}
    \begin{defi}{Scott continuity}
        Let $(D, \sqsubseteq)$ and $(D', \sqsubseteq')$ be complete lattices and $F : D \to D'$ monotonic.
        $F$ is called \emph{continuous} if, for every non-empty chain $S \subseteq D$,
        \begin{tightcenter}
            $F(\bigsqcup S) = \bigsqcup F(S)$.
        \end{tightcenter}
    \end{defi}
\end{halfboxr}
\medskip

\begin{theo}{Kleene's fixpoint theorem}
    Let $(D, \sqsubseteq)$ be a complete lattice and $\Phi : D \to D$ continuous. \\
    Then $F$ has a \emph{least fixed point} $\lfp F$ and a greatest fixed point $\gfp F$:
    \[
        \lfp F := \sup_{n \in \nat} F^n(\bot) \quad \text{and} \quad
        \gfp F := \inf_{n \in \nat} F^n(\top)
    \]
    where $F^0(d) = d$ and $F^{n+1}(d) = F(F^n(d))$.
\end{theo}

\section{Probabilistic Weakest Preconditions}

We assume program variables $x \in \Vars$ are in $\rat$.
We denote arithmetic expressions $E$ over program variables and boolean expressions over program variables $G$.

Usually $\mu$ is a \emph{distribution expression} $\mu : \Sigma \to \Dist(\rat)$ and $p : \Sigma \to [0,1] \cap \rat$.

The \emph{expected value} of a random variable $f : X \to \rel$ under distribution $\mu$ is defined by:
\[
    E_\mu(f) = \sum_{x \in X} f(x) \cdot \mu(x) = \int_X f~\mathrm{d}\mu
\]
Note that the expectation below is a random variable, and distinct from an expected value.

\begin{defi}{Predicate}
    A \emph{predicate} $F$ maps program states to Booleans, i.e.\ $F : \mathbb{S} \to \mathbb{B}$.

    Let $\mathbb{P}$ denote the set of all predicates and $F \sqsubseteq G$ iff $F \Rightarrow G$.
\end{defi}

\begin{defi}{Expectation}
    An \emph{expectation} $f$ maps program states to $\relg \cup \Set{\infty}$, i.e.\ $f : \mathbb{S} \to \relg \cup \Set{\infty}$.

    Let $\mathbb{E}$ denote the set of all expectations and
    \begin{tightcenter}
        $f \sqsubseteq g$ \quad if and only if \quad $f(s) \leq g(s)$ for all $s \in \mathbb{S}$.
    \end{tightcenter}
    \bigskip

    \begin{tightcenter}
        \textsc{Operations}
        \footnotesize{}
        \[
            \begin{array}{cccc}
                (\lam{s} k)(k) = 0 & f[x := E](s) = \begin{cases} f(y) & \text{if } x \neq y \\ \sem{E}_s & \text{otherwise} \end{cases} & (c \cdot f)(s) = c \cdot f(s) & (f + g)(s) = f(s) + g(s)
            \end{array}
        \]
    \end{tightcenter}
\end{defi}

$(\mathbb{E}, \sqsubseteq)$ is a complete lattice with the least element $\lam{s} 0 =: \mathbf{0}$.
The supremum of a subset $S \subseteq \mathbb{E}$ is given by $\sup S = \sup_{f \in S} f$.

We define \emph{predicate} and \emph{expectation transformers} as total functions between predicates $\mathbb{P}$ or expectations $\mathbb{E}$ respectively.

\begin{minipage}[t]{0.52\textwidth}
    \vspace{-\baselineskip}
    \begin{defi}{Weakest pre-expectation}
        For probabilistic program $P$ and $e, f \in \mathbb{E}$, the expectation transformer $\wp(P,\cdot) : \mathbb{E} \to \mathbb{E}$ is defined by $\wp(P,f) = e$ iff $e$ maps each initial state $s$ to the expected value of $f$ after executing $P$ on $s$.
        \[
            \wp(P,f) = \lam{s} \int_\mathbb{S} f~\mathrm{d}P_s
        \]
        where $P_s$ is the distribution over the final states (reached on termination of $P$) when executing $P$ on the initial state $s$.
    \end{defi}
\end{minipage}\hfill%
\begin{minipage}[t]{0.47\textwidth}
    \vspace{-\baselineskip}
    \begin{defi}{Weakest liberal pre-expectation}
        The \emph{weakest liberal precondition} is the expected value of $f$ after executing $P$ on $s$ plus the probability that $P$ diverges on $s$.
        \small{}
        \[
            \wlp(P,f) = \lam{s} \int_\mathbb{S} f~\mathrm{d}P_s + \left( 1 - \int_\mathbb{S} 1 \mathrm{d}P_s \right)
        \]
    \end{defi}

    \centering
    \begin{minipage}[t]{0.75\textwidth}
        $\wlp(P, \cdot) : \mathbb{E}_{\leq 1} \to \mathbb{E}_{\leq 1}$ is defined on \emph{bounded expectations}, i.e.\
        \begin{tightcenter}
            $\mathbb{E}_{\leq 1} = \Set{ f \in \mathbb{E} | f \sqsubseteq \mathbf{1} }$.
        \end{tightcenter}
    \end{minipage}
\end{minipage}

\subsection{Expectation Transformer Semantics of pGCL}

\begin{center}
    \small{}
    \renewcommand*{\arraystretch}{1.8}
    \begin{tabular}{L|CC}
        P & \wp(P,f) & \wlp(P,f) \\ \hline
        \stmtSkip & \multicolumn{2}{C}{f} \\
        \stmtDiverge & 0 & 1\\
        \stmtAsgn{x}{E} & \multicolumn{2}{C}{f[x := E]} \\
        {\color{gray}\stmtObserve{G}} & \multicolumn{2}{C}{\color{gray}[G] \cdot f} \\
        \stmtRasgn{x}{\mu} & \multicolumn{2}{C}{\lam{s} \int_\rat (\lam{v} f(s[x := v]))~\mathrm{d}\mu_s} \\
        \stmtSeq{P_1}{P_2} & \wp(P_1, \wp(P_2, f)) & \wlp(P_1, \wlp(P_2, f)) \\
        \stmtIf{G}{P_1}{P_2} & [G] \cdot \wp(P_1, f) + [\neg G] \cdot \wp(P_2, f) & [G] \cdot \wlp(P_1, f) + [\neg G] \cdot \wlp(P_2, f) \\
        \stmtProb{p}{P_1}{P_2} & p \cdot \wp(P_1, f) + (1-p) \cdot \wp(P_2, f) & p \cdot \wlp(P_1, f) + (1-p) \cdot \wlp(P_2, f) \\
        \stmtWhile{G}{P} & \lfp X.~ ([G] \cdot \wp(P,X) + [\neg G] \cdot f) & \gfp X.~ ([G] \cdot \wlp(P,X) + [\neg G] \cdot f)
    \end{tabular}
\end{center}
\vspace{3\baselineskip}

\begin{halfboxl}
    \vspace{-\baselineskip}
    \begin{theo}{Properties of $\wp$}
        \begin{itemize}[leftmargin=*]
            \item \textsc{Continuity}: \\ \hspace*{1em}
                $\wp(P, \cdot)$ is continuous on $(\mathbb{E}, \sqsubseteq)$.
            \item \textsc{Monotonicity}: \\ \hspace*{1em}
                $f \leq g$ implies $\wp(P,f) \leq \wp(P,g)$.
            \item \textsc{Feasibility}: \\ \hspace*{1em}
                $f \leq \mathbf{k}$ implies $\wp(P,f) \leq \mathbf{k}$.
            \item \textsc{Linearity}: $\Forall r \in \relg$ \\ \hspace*{0.5em}
                {\small{}$\wp(P, r \cdot f + g) = r \cdot \wp(P,f) + \wp(P,g)$}.
            \item \textsc{Strictness}: \\ \hspace*{1em}
                $\wp(P,\mathbf{0}) = \mathbf{0}$.
        \end{itemize}
    \end{theo}

    \sffamily
    {\color{red}Warning!} Not all properties hold for programs with $\stmtObserve{G}$. \\
    {\small{}E.g.\ co-strictness: $\wlp(\stmtObserve{\texttt{false}},~\mathbf{1}) = \mathbf{0}$.}
\end{halfboxl}%
\begin{halfboxr}
    \vspace{-\baselineskip}
    \begin{theo}{Properties of $\wlp$}
        \begin{itemize}[leftmargin=*]
            \item \textsc{Continuity}: \\ \hspace*{1em}
                $\wlp(P, \cdot)$ is continuous on $(\mathbb{E}_{\leq 1}, \sqsubseteq)$.
            \item \textsc{Monotonicity}: \\ \hspace*{1em}
                $f \leq g$ implies $\wlp(P,f) \leq \wlp(P,g)$.
            \item \textsc{Superlinearity}: $\Forall r \in \relg$ \\ \hspace*{0.5em}
                {\small{}$\wlp(P, r \cdot f + g) \leq r \cdot \wlp(P,f) + \wlp(P,g)$}.
            \item \textsc{Duality}: \\ \hspace*{1em}
                {\small{}$\wlp(P,f) = \wp(P,f) + (1-\wp(P,\mathbf{1}))$}
            \item \textsc{Coincidence}: {\small{}for a.s.-terminating $P$} \\ \hspace*{1em}
                $\wlp(P,f) = \wp(P,f)$
            \item \textsc{Co-strictness}: \\ \hspace*{1em}
                $\wlp(P,\mathbf{1}) = \mathbf{1}$.
        \end{itemize}
    \end{theo}
\end{halfboxr}

$\wp(P,\mathbf{1}) = $ termination probability of program $P$.

Using Kleene's Fixpoint Theorem, we can calculate the fixed points for the loops. \\
For $\lfp \Phi = \sup_{n \in \nat} \Phi^n(\bot)$ and for $\gfp \Psi = \inf_{n \in \nat} \Psi^n(\top)$.


\section{Loops and Proof Rules} \label{sec:loops}

Reasoning about loops is the hardest task in program verification.
The weakest preconditions of loops are defined as fixed points and can be approximated iteratively.
But recognizing patterns to yield a closed formula and finding its fixed point is undecidable.
We try to capture the effect of a loop by using a \emph{loop invariant}.

We summarise the results from this section in the table below.
\begin{center}
    \begin{tabular}{c|c|c}
               & \textbf{lower bounds} & \textbf{upper bounds} \\ \hline
        $\wp$  & $\wp$-$\omega$-subinvariant & $\wp$-superinvariant \\
        $\wlp$ & $\wlp$-subinvariant & $\wlp$-$\omega$-superinvariant
    \end{tabular}
\end{center}

\subsection{Predicate Invariants}

\begin{halfboxl}
    We start with non-probabilistic loop invariants.
    A non-probabilistic loop invariant $I$ for a postcondition $F$ is a predicate that holds whenever the loop guard holds, that establishes the postcondition after termination, and also holds during execution of the loop body.
\end{halfboxl}%
\begin{halfboxr}
    \vspace{-\baselineskip}
    \begin{defi}{Loop invariant}
        A predicate $I \in \mathbb{P}$ is a loop invariant if:
        \begin{itemize}
            \item $G \Rightarrow I$,
            \item $\neg G \land I \Rightarrow F$, and
            \item $G \land I \Rightarrow \wlp(P,I)$.
        \end{itemize}
    \end{defi}
\end{halfboxr}

\begin{halfboxl}
    Why does this definition make sense?
    The theorem on the right assures us that a predicate invariant is always a sound approximation of the loop.
    We have shown this directly for probabilistic programs with (non-probabilistic) predicate invaraints.

    $\Psi_{[F]}$ is the wlp-characteristic function of the probabilistic loop for postcondition $[F]$.
\end{halfboxl}%
\begin{halfboxr}
    \vspace{-\baselineskip}
    \begin{theo}{Invariants make sense}
        For $I,F \in \mathbb{P}$ and probabilistic loop $\stmtWhile{G}{P}$ it holds:
        \begin{align*}
            \neg G \land I \Rightarrow F &\text{ and } G \land I \Rightarrow wlp(P,I) \\
            &\text{\large{}iff} \\
            [I] &\sqsubseteq \Psi_{[F]}([I])
        \end{align*}
    \end{theo}
\end{halfboxr}
\vspace{-\baselineskip}

\subsection{Probabilistic Invariants}

Invariants that are not predicates are a bit harder: We use super- and subinvariants.
We need this distinction because only $\wp$-superinvariants are useful for upper bounds for $\wp$, and $\wlp$-subinvariants are useful for lower bounds for $\wlp$.
The other bounds require the respective $\omega$-invariants.

\begin{defi}{Probabilistic invariants}
    Let $\Phi_f$ be the wp-characteristic function of $P'=\stmtWhile{G}{P}$ with respect to post-expectation $f \in \mathbb{E}$ and let $I \in \mathbb{E}$.
    \begin{itemize}
        \item $I$ is a \emph{wp-superinvariant} of $P'$ w.r.t.\ $f$ iff $\Phi_f(I) \leq I$.
        \item $I$ is a \emph{wp-subinvariant} of $P'$ w.r.t.\ $f$ iff $I \leq \Phi_f(I)$.
    \end{itemize}
    Analogously defined for wlp with bounded expectations $\mathbb{E}_{\leq 1}$: Replace $\Phi_f$ by $\Psi_f$.
\end{defi}

\begin{halfboxl}
    With Park's lemma, we can do (co-)induction to find bounds on weakest pre-expectations.
    Induction gives us \textbf{upper bounds for wp}:
    \[
        \underbrace{\Phi_f(I) \leq I}_\text{\tiny{}wp-superinvariant} \text{ implies } \wp(\stmtWhile{G}{P},f) \leq I
    \]
    Co-induction gives us \textbf{lower bounds for wlp}:
    \[
        \underbrace{I \leq \Psi_f(I)}_\text{\tiny{}wlp-subinvariant} \text{ implies } I \leq \wlp(\stmtWhile{G}{P},f)
    \]
\end{halfboxl}%
\begin{halfboxr}
    \vspace{-\baselineskip}
    \begin{theo}{Park's lemma}
        Let $(D,\sqsubseteq)$ be a complete lattice and $\Phi : D \to D$ continuous. Then:
        \begin{align*}
            \Forall d \in D.~ \Phi(d) \sqsubseteq d &\quad\text{ implies }\quad \lfp \Phi \sqsubseteq d \\
            \Forall d \in D.~ d \sqsubseteq \Phi(d) &\quad\text{ implies }\quad d \sqsubseteq \gfp \Phi
        \end{align*}
    \end{theo}
    \centering
    \begin{minipage}[t]{0.8\textwidth}
        \footnotesize{}
        Note that in general, versions of the equations above with $\sqsubseteq \lfp$ and $\gfp \sqsubseteq$ are not valid!
    \end{minipage}
\end{halfboxr}

We can verify a loop invariant $I$ by pushing it through the characteristic function of the loop once, i.e.\ $\Phi(I)$.
For induction, we then only need to verify that $\Phi(I) \sqsubseteq I$, and for co-induction $I \sqsubseteq \Phi(I)$.

\subsection[omega-invariants]{$\omega$-invariants}

$\omega$-invariants give us the missing two bounds we did not get from Park's lemma.
\textbf{$\wp$-$\omega$-subinvariants give lower bounds for $\wp$, and $\wlp$-$\omega$-superinvariants give upper bounds for $\wlp$}.

\begin{defi}{$\omega$-invariants}
    Let $n \in \nat, f \in \mathbb{E}$ and $\Phi_f$ be the $\wp$-characteristic function of the loop $\stmtWhile{G}{P}$.

    Monotonically increasing ($\sqsubseteq$) sequence $(I)_{n \in \nat}$ is a \emph{$\wp$-$\omega$-subinvariant} of the loop w.r.t.\ $f$ iff
    \[
        I_0 \leq \Phi_f(0) \text{ and } I_{n+1} \leq \Phi_f(I_n) \qquad\text{ for all } n \in \nat.
    \]

    \emph{$\wlp$-$\omega$-superinvariants} are defined similarly, where $(I)_{n \in \nat} \in \mathbb{E}_{\leq 1}^\nat$ is monotonically decreasing and $\Phi_f$ is replaced by $\Psi_f$.
\end{defi}

As before, we have two soundness statements for $\omega$-invariants.

\begin{theo}{Bounds on loops using $\omega$-invariants}
    \begin{enumerate}
        \item Let $(I)_{n \in \nat}$ be a $\wp$-$\omega$-subinvariant of $\stmtWhile{G}{P}$ w.r.t. $f \in \mathbb{E}$. Then:
        $$\sup_{n \in \nat} I_n \leq \wp(\stmtWhile{G}{P},f)$$
        \item Let $(I)_{n \in \nat}$ be a $\wlp$-$\omega$-superinvariant of $\stmtWhile{G}{P}$ w.r.t. $f$. Then:
        $$\wlp(\stmtWhile{G}{P},f) \leq \inf_{n \in \nat} I_n$$
    \end{enumerate}
\end{theo}

To verify loops using $\omega$-invariants, the following procedure can be used:
\begin{enumerate}
    \item Find an appropriate $\omega$-invariant $(I)_{n \in \nat}$.
    \item Check that $(I)_{n \in \nat}$ is indeed an $\omega$-invariant:
    \begin{enumerate}
        \item Push $I_n$ through the characteristic function
        \item Check whether this took us above $I_{n+1}$ (for $\wp$) or below $I_{n+1}$ (for $\wlp$) in the partial order $\leq$.
    \end{enumerate}
    \item Find the supremum (for $\wp$) or the infimum (for $\wlp$) of $(I)_{n \in \nat}$ as a lower bound respective upper bound for $\wp$/$\wlp$.
\end{enumerate}

% TODO: proof rules for loops

\iffalse
\begin{defi}{Bayes' Rule}
    The Bayes rule can be used to calculate the probability (\emph{posterior}) of an hypothesis $H$ given some observed evidence $e$:
    $$P(H|e)=\frac{P(e|H) \cdot P(H)}{P(e)}$$
    where
    \begin{itemize}
        \item $P(e|H)$ is the \emph{likelihood}, how probable is the evidence assuming our hypothesis is true,
        \item $P(H)$ is the \emph{prior}, how probable was our hypothesis before observing the evidence, and
        \item $P(e)$ is the \emph{marginal}, how probable is the new evidence under all possible hypothesis.
    \end{itemize}
\end{defi}
\fi

\section{Conditioning}
\label{sec:cpGCL}

We now add $\stmtObserve{G}$ statements.
The idea is that only executions that satisfy all $\stmtObserve{G}$ during execution contribute to the resulting probability distribution.
For this, we need new rules to the operational semantics to pGCL (already on \cpageref{def:pgcl-semantics}).

\subsection{Conditional Expectations}

Then we need normalisation: We want to divide by the probability of not satisfying the observe statements.
We define $\cwp$: as a tuple $(f,g)$.
The intuitive interpretation is that the resulting expected value is given by $f/g$.

\begin{halfboxl}
    \vspace{-\baselineskip}
    \begin{defi}{Conditional expectation}
        A \emph{conditional expectation} is a pair $(f,g)$ with expectation $f \in \mathbb{E}$ and bounded expectation $g \in \mathbb{E}_{\leq 1}$.

        Let $\mathbb{C}=\mathbb{E} \times \mathbb{E}_{\leq 1}$ denote the set of conditional expectations.
        \[
          (f,g) \unlhd (f',g') \text{ iff } f \leq f' \land g \geq g'
        \]
    \end{defi}

    While we have explicit semantics for $\cwp$, it's simplest to just always use the \textsc{Decoupling} property and calculate $\wp$ and $\wlp$ separately.
\end{halfboxl}%
\begin{halfboxr}
    \vspace{-\baselineskip}
    \begin{theo}{Properties of $\cwp$}
        Let $z,z'=(f,g),(f',g') \in \mathbb{C}$.
        \bigskip
        \begin{itemize}[leftmargin=*]
            \item {\large{}\textsc{Decoupling}}: \\[0.5em]
                $\cwp(P,(f,g)) = (\wp(P,f),\wlp(P,g))$.
                \bigskip\footnotesize{}
            \item \textsc{Continuity}: \\ \hspace*{1em}
                $\cwp(P,z)$ is continuous on $(\mathbb{C}, \unlhd)$.
            \item \textsc{Monotonicity}: \\ \hspace*{1em}
                $z \unlhd z'$ implies $\cwp(P,z) \unlhd \cwp(P,z')$.

            \item \textsc{Linearity}: $\Forall r \in \relg$ \\ \hspace*{0.5em}
                {\tiny{}$\cwp(P,(r \cdot f + g,g')) = (r \cdot \wp(P,f) + \wp(P,g),\wlp(P,g'))$.}
            \item \textsc{Strictness}: \\ \hspace*{1em}
                $\cwp(P,(\mathbf{0},\mathbf{1})) = (\mathbf{0},g)$ where $g=\wlp(P,1)$.
            \item \textsc{Feasibility}: {\tiny{} If $\Forall s \in \mathbb{S}.~ g(s) > 0 \Rightarrow f(s)/g(s)$ exists,} \\ \hspace*{1em}
                {\small{}$\Forall s \in \mathbb{S}.~ g'(s) = 0 \text{ implies } f'(s) = 0$} \\ \hspace*{1.5em}
                {\footnotesize{}where $\cwp(P, (f,g)) = (f',g')$.}
        \end{itemize}
    \end{theo}
\end{halfboxr}

\iffalse
\begin{center}
    \small{}
    \renewcommand*{\arraystretch}{1.8}
    \begin{tabular}{L|C}
        P & \cwp(P,(f,g)) \\ \hline
        \stmtSkip &  (f,g) \\
        \stmtDiverge &  (0,1) \\
        \stmtAsgn{x}{E} & (f,g)(\stmtAsgn{x}{E})=(f(\stmtAsgn{x}{E}),g(\stmtAsgn{x}{E})) \\
        \stmtObserve{G} & [G]\cdot (f,g) \\
        \stmtRasgn{x}{\mu} & (\lam{s} \int_\rat (\lam{v} f(s[x := v]))~\mathrm{d}\mu_s, ...g)\\
        \stmtSeq{P_1}{P_2} & \cwp(P_1,\cwp(P_2,(f,g))) \\
        \stmtIf{G}{P_1}{P_2} & [G] \cdot \cwp(P_1,(f,g)) + [\neg G] \cdot \cwp(P_2,(f,g)) \\
        \stmtProb{p}{P_1}{P_2} & p \cdot \cwp(P_1,(f,g)) + (1-p) \cdot \cwp(P_2,(f,g)) \\
        \stmtWhile{G}{P} & \lfp_{\unlhd}(X,Y). [\neg G] \cdot (f,g) + [G] \cdot \cwp(P,(X,Y))
    \end{tabular}
\end{center}
\vspace{3\baselineskip}
\fi

\subsection{Program Transformations}

It turns out $\stmtObserve{G}$ is entirely syntactic sugar: We can transform a program with $\stmtObserve{G}$ statements to one without while preserving semantics.

\subsubsection{Rejection Sampling}

The idea is to restart an infeasable run until all $\stmtObserve{G}$ statements are fulfilled.

We introduce a \texttt{flag} variable to signal violation of an $\stmtObserve{G}$ and new variables $sx_i$ for every variable $x_i$ in the original program.
The $sx_i$ variables are used to reset the variables $x_i$ in case an $\stmtObserve{G}$ is violated and we need to restart.
Initially store the value of $x_1$ in $xs_1$ and reset $x_1$ to $xs_1$ in the beginning of every loop iteration.

\begin{minipage}[t]{0.55\textwidth}
    We modify the original program $prog$ to $mprog$:
    \begin{align*}
        \stmtObserve{G} &\rightsquigarrow flag:=!G \mid \mid flag \\
        \stmtDiverge &\rightsquigarrow \stmtIf{!flag}{abort}{} \\
        \stmtWhile{G}{prog} &\rightsquigarrow \stmtWhile{G \text{ \&\& } !flag}{prog}
    \end{align*}
\end{minipage}\hfill%
\begin{minipage}[t]{0.4\textwidth}
    \small{}
    The result is something like:
    \begin{lstlisting}
    $sx_1$,... := $x_1$,...;
    flag := true;
    $\texttt{while}(flag)\{$
        $flag:=false;$
        $x_1$,... := $sx_1$,...;
        mprog;
    }
    \end{lstlisting}
\end{minipage}

This transformation is correct: For a cpGCL program $P$ and $\hat{P}$ the result of the above transformation we have:
\[
    \cwp(P,(f,1))=\wp(\hat{P},f)
\]

\begin{halfboxl}
    We can also go the other way: If a loop is \emph{iid}, then
    $\cwp(\texttt{repeat } P \texttt{ until (}G\texttt{)}, (f,g))$
    equals
    $\cwp(\stmtSeq{P}{\stmtObserve{G}},(f,g))$.

    \emph{iid} means that $G$ holds after $P$ independently of the expected value of $f$ after $P$.
\end{halfboxl}%
\begin{halfboxr}
    \vspace{-\baselineskip}
    \begin{defi}{iid-loop}
        A loop $\stmtWhile{G}{P}$ is \emph{iid} iff for any expectation $f$:
        \small{}
        \[
            \wp(P, [G] \cdot \wp(P,f)) = \wp(P, [G]) \cdot \wp(P, f).
        \]
    \end{defi}
\end{halfboxr}

\vspace{-\baselineskip}

\subsubsection{Hoisting}

A second option to deal with $\texttt{observe}$-statements is to use \emph{hoisting}.
It's a wee bit more complicated and the correctness theorem only holds for programs with at least one feasible run.
Hoisting removes the \texttt{observe}-statements and transforms the probabilities accordingly.

\begin{defi}{Hoisting}
    \begin{align*}
        T(\stmtSkip,f) &= (\stmtSkip,f)\\
        T(\stmtDiverge,f) &= (\stmtDiverge,1)\\
        T(\stmtAsgn{x}{E},f) &= (\stmtAsgn{x}{E},f[\stmtAsgn{x}{E}]) \\
        T(\stmtObserve{G},f) &= (\stmtSkip,[G] \cdot f) \\
        T(\stmtSeq{P_1}{P_2},f) &= (\stmtSeq{Q_1}{Q_2},h) \text{ where } (Q_2,g)=T(P_2,f) \text{ and } (Q_1,h)=T(P_1,g) \\
        T(\stmtIf{G}{P_1}{P_2},f) &= (\stmtIf{G}{Q_1}{Q_2},[G] \cdot g + [\neg G] \cdot h) \\
        &  \text{where } (Q_1,g)=T(P_1,f) \text{ and } (Q_2,h)=T(P_2,f)\\
        T(\stmtProb{p}{P_1}{P_2},f) &= (\stmtProb{q}{Q_1}{Q_2},p \cdot g + (1-p) \cdot h) \text{ where } q=\frac{p \cdot g}{p \cdot g + (1-p) \cdot h},\\
        & (Q_1,g)=T(P_1,f) \text{ and } (Q_2,h)=T(P_2,f) \\
        T(\stmtWhile{G}{P},f) &= (\stmtWhile{G}{Q},g) \text{ where } g= \gfp H \\
        & \text{ with } H(h)=[G] \cdot (\pi_2 \odot T)(P,h) + [\neg G] \cdot f \text{ and } (Q, \cdot)=T(P,g)\\
    \end{align*}
\end{defi}

\begin{theo}{Correctness of hoisting}
    For any cpGCL program $P$ with at least one feasible run and $f \in \mathbb{E}$:
    $$\cwp(P,(f,1)) = \wp(Q,f) \text{  with  } T(P,1)=(Q,h).$$
    The component $h$ represents the probability that $P$ satisfies all its \texttt{observe}-statements.
\end{theo}

\section{Arithmetical Hierarchy}
In this section, we rank different undecidable decision problems.
Although each of them is undecidable, some are still harder than others.
The \emph{arithmetical hierarchy} classifies decision problems by the complexity of characterising formulas in first-order Peano arithmetic.

\begin{defi}{Arithmetical hierarchy}
    The \emph{arithmetical hierarchy} consists of three types of classes: $\Sigma_n,\Pi_n$ and $\Delta_n$ for each $n \in \nat$.

    Classes $\Sigma_n$, where $R$ is a decidable relation:
    $$\Sigma_n=\{A \mid A = \{ x \mid \exists y_1 \forall y_2 \exists y_3 ... \forall/\exists y_n : (x,y_1,...,y_n) \in R \} \}.$$


    Classes $\Pi_n$, where $R$ is a decidable relation:
    $$\Pi_n=\{A \mid A = \{ x \mid \forall y_1 \exists y_2 \forall y_3 ... \forall/\exists y_n : (x,y_1,...,y_n) \in R \} \}$$

    Classes $\Delta_n$ are defined as $\Delta_n = \Sigma_n \cap \Pi_n$.
\end{defi}

\begin{theo}{Elementary properties of the arithmetical hierarchy}
    \begin{itemize}[leftmargin=*]
        \item $\Delta_1$ is the class of decidable problems.
        \item Classes $\Sigma_n, \Pi_n$ and $\Delta_n$ are closed under conjunction and disjunction.
        \item $\Delta_n$ is closed under negation.
        \item The classes $\Sigma_n$ and $\Pi_n$ are complementary.
        \item There is a strict relation between classes:
            $\Sigma_n \subset \Delta_{n+1} \subset \Pi_{n+1}$ and $\Pi_n \subset \Delta_{n+1} \subset \Sigma_{n+1}$.
        \item If problem $A$ is $\Sigma_n$-complete, then its complement is $\Pi_n$-complete, and vice versa.
    \end{itemize}
\end{theo}

\begin{halfboxl}
    As with polynomial complexities, we also have a notion of \emph{completeness} of a problem in a certain complexity class here.

    By \emph{Davis' theorem}, we know that if problem $A$ is    $\Sigma_n$-complete, then $A \in \Sigma_n \setminus \Pi_n$, and vice versa.

    Some simple examples for well-known decision problems follow.
\end{halfboxl}%
\begin{halfboxr}
    \vspace{-\baselineskip}
    \begin{defi}{Reducibility and completeness}
        $A \subseteq X$ is \emph{reducible} to $B \subseteq X$ if there is a computable function $f : X \to X$ such that
        \[
            \Forall x \in X.~ x \in A \text{ iff } f(x) \in B.
        \]
        Decision problem $A$ is \emph{$\Gamma_n$-hard} iff every $B \in \Gamma_n$ can be reduced to $A$.

        $A$ is \emph{$\Gamma_n$-complete} iff $A \in \Gamma_n$ and $A$ is $\Gamma_n$-hard.
    \end{defi}
\end{halfboxr}

\paragraph{Halting problem} ($H \in \Sigma_1$, $\Sigma_1$-complete) \quad Program $P$, state $s$. $(P,s) \in H$ iff:
\smallskip
\begin{tightcenter}$
    \exists k \in \nat, s' \in \mathbb{S}. P \text{ terminates on input } s \text{ in } k \text{ steps in state }s'.
$\end{tightcenter}

\paragraph{Universal halting problem} ($UH \in \Pi_2$, $\Pi_2$-complete) \quad Program $P$, state $s$. $P \in UH$ iff:
\smallskip
\begin{tightcenter}$
    \forall s\in \mathbb{S}. (P,s) \in H.
$\end{tightcenter}

\paragraph{Co-finiteness problem} ($COF \in \Sigma_3$, $\Sigma_3$-complete) \quad Program $P$. $P \in COF$ iff:
\smallskip
\begin{tightcenter}$
    \Set{ s \in \mathbb{S} | (P,s) \in H } \text{ is co-finite}
$\end{tightcenter}
where a subset $A$ of $X$ is \emph{co-finite} if $X \setminus A$ is finite.
\bigskip

\begin{halfboxl}
    The important results are about the decision problems $LEXP$, $REXP$, $EXP$ and $FEXP$.

    $LEXP$ is $\Sigma_1$-complete:
    \begin{tightcenter}
        $\exists y.\ q < \sum_{k=0}^y \wp^{=k} (P,f)(s)$
    \end{tightcenter}
    \medskip

    $REXP$ is $\Sigma_2$-complete:
    \begin{tightcenter}
        $\exists \delta >0.\ \forall y: q-\delta > \sum_{k=0}^y \wp^{=k}(P,f)(s)$
    \end{tightcenter}
    \medskip

    $EXP$ is $\Pi_2$-complete.
    $FEXP$ is $\Sigma_2$-complete.

    {\footnotesize{} Note the $\delta$ in $REXP$ which is (intentionally) missing in $LEXP$.}
\end{halfboxl}%
\begin{halfboxr}
    \vspace{-\baselineskip}
    \begin{defi}{The decision problems $LEXP$, $REXP$, $EXP$ and $FEXP$}
        Let $P$ be a pGCL program, $s \in \mathbb{S}$ a variable solution, $q \in \rat_{\geq 0}$ and $f: \mathbb{S} \to \rat_{\geq 0}$ a computable function. Then:
        \begin{align*}
            (P,s,f,q) \in LEXP &\text{ iff } q < \wp(P,f)(s) \\
            (P,s,f,q) \in REXP &\text{ iff } q > \wp(P,f)(s) \\
            (P,s,f,q) \in EXP &\text{ iff } q = \wp(P,f)(s) \\
            (P,s,f) \in FEXP &\text{ iff } \wp(P,f)(s) < \infty
        \end{align*}
    \end{defi}
\end{halfboxr}

\section{Almost-Sure Termination}

\begin{halfboxl}
    With probabilistic programs, termination has a few different degrees:
    \begin{itemize}[leftmargin=*]
        \item \emph{Certain termination}: Literally every single program execution terminates.
        \item \emph{Almost-sure termination}: Termination with probability one, but there may still be runs with infinite runtime.
        \begin{itemize}
            \item[+] \emph{Positive almost-sure termination}: Expected finite number of steps.
            \item[+] \emph{Null almost-sure termination}: Expected infinite number of steps.
        \end{itemize}
    \end{itemize}

    $AST$ and $UAST$ are both $\Pi_2$-complete.
    $UPAST$ is $\Pi_3$-complete.
    $PAST$ is $\Sigma_2$-complete:
    \begin{tightcenter}
        $\exists c.\ \forall l\ \ert^{\leq l} (P,s) < c$.
    \end{tightcenter}
\end{halfboxl}%
\begin{halfboxr}
    \vspace{-\baselineskip}
    \begin{defi}{Decision problems $AST$ and $UAST$}
        Let $P$ be a program, $s \in \mathbb{S}$ a valuation.
        \begin{align*}
            (P,s) \in AST &\text{ iff } \wp(P,1)(s)=1 \\
            P \in UAST &\text{ iff } \forall s \in \mathbb{S}. (P,s) \in AST
        \end{align*}
    \end{defi}

    \begin{defi}{Decision problems $PAST$ and $UPAST$}
        Let $P$ be a program, $s \in \mathbb{S}$ a valuation.
        \begin{align*}
            (P,s) \in PAST &\text{ iff } \ert(P,s) < \infty \\
            P \in UPAST &\text{ iff } \forall s \in \mathbb{S}. (P,s) \in PAST
        \end{align*}
    \end{defi}
\end{halfboxr}

\subsection{Proving Termination}

\begin{halfboxl}
    Our aim is to prove termination by using a \emph{variant function} (or \emph{ranking function}) for the state space of a program that is monotonically decreasing in every loop iteration.
    The function deacreases with respect to a (strict) well-founded relation.

    We know that every universally terminating (non-probabilistic) loop $\stmtWhile{G}{P}$ has a variant function.

    Below are two theorems: The first is to prove positive almost-sure termination using a \emph{ranking super-invariant}.
    The second theorem is for (positive/null) almost-sure termination and requires a variant $I$, a decrease probability function $p$ and a decrease function $d$.
\end{halfboxl}%
\begin{halfboxr}
    \vspace{-\baselineskip}
    \begin{defi}{Well-founded relation}
        Let $(D,\sqsubset)$ be a strict partial order. The relation $\sqsubset$ is well-founded if there is no infinite sequence $d_1,d_2, d_3, ...$ with $d_i \in D$ such that $d_i \sqsubset d_{i+1}$ for all $i \in \nat$.
    \end{defi}

    \begin{defi}{Variant function}
        A \emph{variant} function $V: \mathbb{S} \to \rel$ for GCL-loop $\stmtWhile{G}{P}$ is a function that satisfies for every $s \in \mathbb{S}$:
        \begin{itemize}[leftmargin=*]
            \item If $s \models G$, then the execution of $P$ on $s$ terminates in a state $t$ with:
                $$V(t) \leq V(s) - \varepsilon \text{ for some fixed } \varepsilon > 0$$
            \item If $V(s) \leq 0$, then $s \not\models G$.
        \end{itemize}
    \end{defi}
\end{halfboxr}

\begin{theo}{Proving positive almost-sure termination (PAST)}
    Let $\stmtWhile{G}{P}$ be a loop where $P$ terminates universally certainly (P is loop-free), and let $I \in \mathbb{E}$ be a \emph{ranking super-invariant} of the loop w.r.t. expectation $0$, i.e., $I \leq \infty$ and for some constants $\varepsilon$ and $K$ with $0 < \varepsilon < K$ it holds:
    \begin{enumerate}
        \item $[\neg G] \cdot I \leq K$
        \item $[G]\cdot K \leq [G] \cdot I + [\neg G]$
        \item $\Phi(I) \leq [G] \cdot (I - \varepsilon)$
    \end{enumerate}
    Then: $\stmtWhile{G}{P}$ terminates universally positively almost-surely.
\end{theo}

\begin{theo}{Proof rule for almost-sure termination (AST)}
    Let $I \in \mathbb{P}$, (variant) function $V: \mathbb{S} \to \relg$, (probability) function $p: \relg \to (0,1]$ be antitone, (decrease) function $d: \relg \to \relg$ be antitone. If:
    \begin{enumerate}
        \item $[I]$ is a $\wp$-subinvariant of $\stmtWhile{G}{P}$ w.r.t. $[I]$
        \item $V=0$ indicates termination, i.e. $[\neg G]=[V=0]$
        \item $V$ is a super-invariant of $\stmtWhile{G}{P}$ w.r.t. $V$
        \item $V$ satisfies the progress condition
            $$p \circ (V \cdot [G] \cdot [I]) \leq \lam{s}\wp(P,[V \leq V(s) -d(V(s))])(s)$$
    \end{enumerate}
    Then: The loop $\stmtWhile{G}{P}$ terminates from any state $s$ satisfying the invariant $I$, i.e.,
        $$[I] \leq \wp(\stmtWhile{G}{P},1)$$
\end{theo}

\section{Expected Runtimes}

\begin{halfboxl}
    \vspace{-\baselineskip}
    \begin{defi}{Runtimes}
        A runtime $t: \mathbb{S} \to \relg \cup \Set{ \infty } $.

        Let $\mathbb{T}$ denote the set of all runtimes.
    \end{defi}

    In our time model we use a single time unit for $skip$, any assignment, evaluating a guard or probabilistic choice.
    Sequential composition does not take any time.

    For every pGCL program $P$ and input state $s$:
    \[
        \underbrace{\ert(P, \mathbf{0})(s) < \infty}_\text{\tiny{}positive a.s.-termination on $s$} \text{ implies } \underbrace{\wp(P,\mathbf{1})(s) = \mathbf{1}}_\text{\tiny{}a.s.-termination on $s$}
    \]
    And:
    \[
        \underbrace{\ert(P,\mathbf{0}) < \infty}_\text{\tiny{}universal positive a.s.-termination} \text{ implies } \underbrace{\wp(P,\mathbf{1}) = \mathbf{1}}_\text{\tiny{}universal a.s.-termination}
    \]
\end{halfboxl}%
\begin{halfboxr}
    \vspace{-\baselineskip}
    \begin{theo}{Properties of $\cwp$}
        \begin{itemize}[leftmargin=*]
            \item \textsc{Continuity}: \\ \hspace*{1em}
                $\ert(P,t)$ is continuous on $(\mathbb{T}, \leq)$.
            \item \textsc{Monotonicity}: \\ \hspace*{1em}
                $t \leq t'$ implies $\ert(P,t) \leq \ert(P,t')$.
            \item \textsc{Constant propagation}: \\ \hspace*{1em}
                $\ert(P,k+t)=k+\ert(P,t)$.
            \item \textsc{Preservation of $\infty$}: \\ \hspace*{1em}
                $\ert(P, \infty)=\infty$.
            \item \textsc{Connection to wp}: \\ \hspace*{1em}
                $\ert(P,t)=\ert(P,0)+\wp(P,t)$.
            \item \textsc{Affinity}: \\
                {\scriptsize{}$\ert(P, a \cdot t + t') = \ert(P,0)+ a \cdot \ert(P,t) + \ert(P,t') $.}
        \end{itemize}
    \end{theo}
\end{halfboxr}

\begin{center}
    \small{}
    \renewcommand*{\arraystretch}{1.8}
    \begin{tabular}{L|C}
        P & \ert(P,t) \\ \hline
        \stmtSkip & 1+t \\
        \stmtDiverge &  \infty \\
        \stmtAsgn{x}{E} & 1+t[x:=E] \\
        \stmtRasgn{x}{\mu} & 1+\lam{s} \int_\rat (\lam{v}t(s[x := v]))~\mathrm{d}\mu_s\\
        \stmtSeq{P_1}{P_2} & \ert(P_1,\ert(P_2,t)) \\
        \stmtIf{G}{P_1}{P_2} & 1+ [G] \cdot \ert(P_1,t + [\neg G] \cdot \ert(P_2,t) \\
        \stmtProb{p}{P_1}{P_2} & 1+ p \cdot \ert(P_1,t) + (1-p) \cdot \ert(P_2,t) \\
        \stmtWhile{G}{P} & \lfp X. (1+ [G] \cdot \ert(P,X) +[\neg G] \cdot t)
    \end{tabular}
\end{center}
\vspace{2\baselineskip}

We can also add rewards corresponding to the runtimes to the Markov chain for a program.
State $\angled{\down,s}$ gets reward $t(s)$.
State $\angled{\stmtDiverge,s}$ gets reward $\infty$.
State $\angled{\stmtSeq{P_1}{P_2},s}$ gets reward $0$.
All other states get reward $1$.
Then: $\ert(P,0)(s) = ER^{\sem{P}}(s, \diamond sink)$.

Using the $\cwp$-calculus, we were able to show that $PAST$ is not compositional, i.e.\ for programs $P_1$, $P_2$ that are positive a.s.-terminating, $\stmtSeq{P_1}{P_2}$ is not necessarily also positive a.s.-terminating.

Similar to $\wp$-calculus, we can also define \emph{runtime-$\omega$-subinvariants} for lower bounds and \emph{runtime-superinvariants} for upper bounds (see \cref{sec:loops}).

\section{Bayesian Networks}

A more traditional approach to statistical inference are \emph{Bayesian networks}.
A Bayesian network is a directed acyclic graph of nodes representing random variables.
Edges represent causal relationships.
Each random variable has an associated \emph{conditional probability table} that maps all values of a node's $k$ parents to a probability distribution.
If there are no parents, then $\Theta_v : () \to \Dist(D) \equiv \Dist(D)$.

\begin{defi}{Bayesian network}
    A \emph{Bayesian network} (BN) is a tuple $B=(V,E,\Theta)$ where
    \begin{itemize}
        \item $(V,E)$ is a directed acyclic graph with
            \begin{itemize}
                \item finite $V$ in which each $v \in V$ represents a random variable with values from finite domain $D$, and
                \item $(v,w) \in E$ represents the (causal) dependencies of $w$ on $v$.
            \end{itemize}
        \item For each vertex $v$ with $k$ parents, the function $\Theta_v: D^k \to \Dist(D)$ is the \emph{conditional probability table} of (the random variable represented by) vertex $v$.
    \end{itemize}
    $w \in V$ is a \emph{parent} of $v \in V$ whenever $(w,v) \in E$.
\end{defi}

The \emph{joint probability function} of a Bayesian network gives a semantics to the network: By simple recursive multiplication of probabilities, we can calculate any joint probability of variables in the network.

\begin{defi}{Joint probability function of a BN}
    Let $B=(V,E,\Theta)$ be a BN, and $W \subseteq V$ be a downward closed set of vertices where $w \in W$ has value $\underline{w} \in D$. The (unique) \emph{joint probability} function of BN $B$ in which the nodes in $W$ have values $\underline{W}$ equals:
    $$\Pr(W=\underline{W})= \prod_{w \in W} \Pr(w = \underline{w} \mid parents(w)=\underline{parents(w)})=\prod_{w \in W} \Theta_w(\underline{parents(w)})(\underline{w}).$$
    The \emph{conditional probability distribution} of $W \subseteq V$ given observations on a set $O \sqsubseteq V$ is given by $$\Pr(W= \underline{W} \mid O = \underline{O})=\frac{\Pr(W= \underline{W} \land O = \underline{O})}{\Pr(O= \underline{O})}.$$
\end{defi}

\subsection{Conditional Independence}

Two independent events may become dependent given some observation.

\begin{defi}{Conditional independence}
    Let $X,Y,Z$ be (discrete) random variables. $X$ is \emph{conditionally independent} of $Y$ given $Z$, denoted $I(X,Z,Y)$, whenever:
    $$\Pr(X \land Y \mid Z)=\Pr(X \mid Z) \cdot \Pr(Y \mid Z) \text{ or } \Pr(Z)=0.$$
\end{defi}

\begin{theo}{Graphoid axioms}
    Conditional independence satisfies the following axioms for disjoint sets of random variables $W,X,Y,Z$:
    \begin{enumerate}
        \item Symmetry: $I(X,Z,Y)$ iff $I(Y,Z,X)$
        \item Decomposition: $I(X,Z,Y \cup W)$ implies ($I(X,Z,Y)$ and $I(X,Z,W))$
        \item Weak union: $I(X,Z,Y \cup W)$ implies $I(X,Z \cup Y,W)$
        \item Contraction: ($I(X,Z,Y)$ and  $I(X,Z \cup Y,W)$) implies $I(X,Z,Y \cup W)$
        \item Triviality: $I(X,Z,\emptyset)$
    \end{enumerate}
\end{theo}

\begin{halfboxl}
    \emph{D-separation}\footnote{see also our \href{https://panikzettel.philworld.de/ai.pdf}{Artifical Intelligence Panikzettel} for another take at d-separation.} is a sufficient condition for conditional independence.
    Define all undirect paths in the DAG of the BN as a \emph{pipe} and every vertex on a path as a \emph{valve}.
    Valves are either \emph{open} or \emph{closed}.
    A pipe is \emph{blocked} if at least one valve on the path is closed.

    A valve $v$ is closed for a variable set $Z$:
    \begin{enumerate}
        \item \emph{Sequential}: $v \in Z$ is a child of one neighbour and a parent of the other neighbour.
        \item \emph{Divergent}: $v \in Z$ is a parent of both neighbours.
        \item \emph{Convergent}: neither $v$ nor any of its directly reachable descendants are in $Z$.
    \end{enumerate}

    The algorithm on the right is a polynomial time check for d-seperation:
    $dsep_G(X,Y,Z)$ iff $X$ and $Y$ are disconnected in $prune_{X,Y,Z}(G)$.

    And since $dsep_G(X,Y,Z)$ implies $I(X,Y,Z)$, we can sometimes provide a guarantee that sets of nodes are conditionally independent in polynomial time.
\end{halfboxl}%
\begin{halfboxr}
    \vspace{-\baselineskip}
    \begin{defi}{d-seperation}
        Let $X,Y,Z$ be disjoint sets of vertices in the DAG $G$.
        $X$ and $Y$ are \emph{d-seperated} by $Z$ in $G$, denoted $dsep_G(X,Z,Y)$, iff:

        Every (undirected) path between a vertex in $X$ and a vertex in $Y$ is blocked by some vertex in $Z$.
    \end{defi}

    \begin{algo}{d-separation polynomial time}
        \textbf{Input}: DAG $G$ and disjoint sets of vertices $X$, $Y$, $Z$.

        \textbf{Output}: DAG $prune_{X,Y,Z}(G)$.
        \tcblower
        \begin{enumerate}
            \item Repeat as long as possible: \\
                  Eliminate any leaf vertex $v$ from $G$ with $v \notin X \cup Y \cup Z$.
            \item Eliminate all edges emanating from  vertices in $Z$.
            \item Return remaining graph.
        \end{enumerate}
    \end{algo}
\end{halfboxr}

The complexity of inference on a BN is measured in terms of the \emph{Markov blanket}, a degree of dependence in the BN.
The less dependent the BN is the simpler is the probabilistic inference.

\begin{defi}{Markov blanket}
    The Markov blanket for a vertex $v$ in a BN is the set $\partial v$ of vertices composed of $v, v'$s parents, its children, and its children's other parents.

    The average Markov blanket of BN $B$ is the average size of the Markov blanket of all its vertices, that is, $\frac{1}{|V|}\sum_{v \in V} |\partial v|$.
\end{defi}

Every set of vertices in a BN is conditionally independent of $v$ when conditioned on $\partial v$. Thus, for distinct vertices $v$ and $w$:
$$\Pr(v \mid w \land \partial v) = \Pr(v \mid \partial v) \text{ which is equivalent to } I(\{v\},\{w\},Z).$$

\subsection{Probabilistic Inference}

\begin{halfboxl}
    The decision problems TI and STI are PP-complete.

    PP (\emph{Probabilistic Polynomial-Time}) is the class of decision problems solvable by a probabilistic Turing machine in polynomial time with an error probability $< \frac{1}{2}$.

    We have shown PP-completeness by reducing MAJSAT, another PP-complete problem, to STI.
    And since STI is a special case of TI, MAJSAT can also be reduced to STI.
\end{halfboxl}%
\begin{halfboxr}
    \vspace{-\baselineskip}
    \begin{defi}{Probabilistic inference problems}
        Let $B$ be a BN with set $V$ of vertices, the \emph{evidence} $E \subseteq V$ and the \emph{questions} $Q \subseteq V$.

        The \emph{probabilistic inference problem} is to determine the conditional probability:
        $$\Pr(Q=q \mid E=e)=\frac{\Pr(Q=q \land E=e)}{\Pr(E=e)}$$

        Variants for probability $p \in \rat \cap [0,1)$:
        \begin{itemize}
            \item \emph{Threshold Inference} (TI): \\
                   Is $\Pr(Q=q \mid E=e) > p$?
            \item \emph{Simple TI} (STI): \\
                  Is $\Pr(E=e) > p$?
        \end{itemize}
    \end{defi}
\end{halfboxr}

BNs correspond to ``simple'' probabilistic programs as there is no ``data-flow'' between loop iterations. Such programs are called iid. If $\stmtWhile{G}{P}$ is iid for expectation $f$, it holds for every state $s$:

\[
    \wp(\stmtWhile{G}{P},f)(s) = [G](s) + \frac{\wp(P, [\neg G] \cdot f)(s)}{1 - \wp(P,[G])(s)} + [\neg G](s) \cdot f(s)
\]
where we let $\frac{0}{0}=0$.

We can also use our $\ert$-calculus to calculate the expected sample time for a BN.
This is very helpful to prove that there is no way the Windows printer troubleshooter will ever return a good result.
Similar to the $\wp$-rule for iid-loops above, we have for \textbf{a.s.-terminating iid loops}:
\[
    \ert(\stmtWhile{G}{P},t) = \mathbf{1} + \frac{\mathbf{1} + \ert(P, [\neg G] \cdot t)}{1 - \wp(P,[G])} + [\neg G](s) \cdot t.
\]

\end{document}
