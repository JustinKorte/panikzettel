\documentclass[11pt]{scrartcl}
\usepackage[utf8]{inputenc}
\usepackage{mathtools}
\usepackage{amssymb}
\usepackage{amsmath}
\usepackage{listings}
\usepackage{xcolor}
\usepackage{hanging}
\usepackage{calc}
\usepackage[ngerman]{babel}
\title{\textbf{Lineare Algebra Panikzettel}}
\author{Philipp Schröer, Tobias Polock}
\date{\today}

% Formatting
\setlength\parindent{0pt}
\relpenalty=9999
\binoppenalty=9999

% Code blocks
\lstnewenvironment{mat}
{\lstset{language=Mathematica,
	mathescape,
    columns=flexible,
    backgroundcolor=\color{lightgray!10},
%    numberstyle=\tiny\color{gray},
%    numbers=left,
    basicstyle=\small\ttfamily}
    }
{}

% Definitions
\newlength{\hangwidth}
\newcommand{\skript}[1]{\settowidth{\hangwidth}{\textbf{(#1)} }\hangpara{\hangwidth}{1}\textbf{(#1)} }

\newcommand{\id}{\mathrm{id}}
\newcommand{\Sol}{\mathrm{Sol}}
\newcommand{\rk}{\mathrm{rk}}
\newcommand{\GL}{\mathrm{GL}}
\newcommand{\Field}{\mathbb{F}}

\DeclarePairedDelimiter\abs{\lvert}{\rvert}


\begin{document}
\maketitle

\section{Vektorräume}

$K$-Vektorraum $V$ hat die Operationen Addition  $V \times V \to V, (v,w) \mapsto v+w$, sowie die Skalarmultiplikation $K \times V \to V, (a,v) \mapsto a \cdot v  $, für den folgende Axiome mit $v, w, x \in V, a,b \in K$ gelten müssen:

\begin{enumerate}
\item Assoziativität d. Addition: $(v+w)+x = v+(w+x)$
\item Nullvektor: $\exists 0 \in V: v+0=0+v=v$
\item Negative: $(-v)+v=v+(-v)=0$
\item Kommutative Additon: $v+w=w+v$
\item Assoziative Skalarmultiplikation: $a(bv)=(ab)v$
\item Einselement: $1\cdot v=v$
\item Distributivität: $(a+b)v=(av)+(bv)$, sowie $a(v+w)=(av)+(aw)$
\end{enumerate}

\section{Matrixkalkül}

\subsection{Wiederholung Diskrete Strukturen}

\subsubsection{Zeilen-/Spaltenoperatoren}

Eine Zeilenoperation ist eine Verkettung von Elementaren Zeilenoperatoren.
Eine Zeilenoperation $z$ hat eine eindeutige Matrix $Z = z(\mathrm{E})$, sodass $z(A) = ZA$. (E ist die Einheitsmatrix).	\\
Diese Matrix nennen wir Zeilenoperator. Der Zeilenoperator $Z$ einer Zeilenoperation $z = z_1 \circ \ldots \circ z_n$ mit Zeilenoperationen $z_1,\ldots,z_n$ ist das Produkt $Z_1 \cdot \ldots \cdot Z_n$ der Zeilenoperatoren zu $z_1,\ldots,z_n$.	\\
Für Spaltenoperationen $z$ sind Spaltenoperatoren $Z = z(\mathrm{E})$ analog, nur dass $z(A) = AZ$ (das Z steht auf der rechten Seite) und dass der Zeilenoperator zu $z_1 \circ \ldots \circ z_n$ $Z_n \cdot \ldots \cdot Z_1$ ist.

\subsection{Spalteninterpretation}

% Größe von B?
Für $m,n \in \mathbb{N}_0, A \in K^{m \times n}$:

\skript{3.1} $\varphi_A: K^{n \times 1} \to K^{m \times 1}, x \mapsto Ax$

\skript{3.4} $\varphi_A(x) = \sum_{j =1}^n  x_j A_{-,j}$

\skript{3.5 a} $\varphi_{BA}=\varphi_B \circ \varphi_A$

\skript{3.5 b} $\varphi_{\mathrm{E}_n} = \id_{K^{n \times 1}}$

\skript{3.5 c} $\varphi_{A}^{-1}=\varphi_{A^{-1}}$

\subsection{Koordinatenspalte}

\skript{3.6}$\kappa_s : V \to K^{n \times 1}, v \mapsto \kappa_s(v)$ \\
ist der zu $K^{n \times 1} \to V, a \mapsto \sum_{i \in [1,n]} a_i s_i$ inverse VR.-Hom.

\textit{Beispiel.} Koordinatenspalte von $a \in K^2$ bzgl. $s = ((1,0), (1,1))$ ist $$\kappa_s(a) = \begin{pmatrix} a_1 - a_2 \\ a_2 \end{pmatrix}$$ denn $a = (a_1, a_2) = (a_1 - a_2) (1,0) + a_2 (1,1)$. Dazu einfach das LGS lösen mit $$\sum_{i \in [1,n]} (\kappa_s(v))_i s_i = v$$ d.h. die Basis spaltenweise in die LGS-Matrix eintragen und lösen.

\begin{mat}
> LinearSolve[{{1, 0}, {1, 1}} // Transpose, {a1, a2}]
{a1-a2,a2}
\end{mat}

\subsection{Darstellungsmatrix}

\skript{3.8}$M_{t,s}(\varphi) = (\kappa_t(\varphi(s_1)) \ldots \kappa_t(\varphi(s_n)))$ \\
für $\varphi : V \to W \in Hom(V, W)$, Basen $s = (s_1, \ldots, s_n)$ von $V$ und $t = (t_1, \ldots, t_m)$ von $W$.

Es muss also für jede Spalte das $s_i$ in $\varphi$ eingesetzt werden und das Ergebnis wieder nach den Linearfaktoren von $t$ aufgelöst werden.

\textit{Beispiel.}
\begin{mat}
> phi[{x1_, x2_}] := {x1+x2, x1-x2, x2}
> s = {{1, 0}, {0, 1}}
> t = {{1,0,0}, {0, 1, 0}, {0, 0, 1}}
> {
	LinearSolve[t // Transpose, phi[s[[1]]]],
	LinearSolve[t // Transpose, phi[s[[2]]]]
  } // Transpose
{
  {1, 1},
  {1, -1},
  {0, 1}
}
\end{mat}

\skript{Tutoraufgabe 30.5.16 a}$M_{e,e}(\varphi_A) = A$

\subsection{Basiswechselmatrix, Basiswechselformeln}

\skript{3.15}In einem Vektorraum $V$ mit Basen $s = (s_1,\ldots, s_n)$ und $s^\prime = (s^\prime_1, \ldots, s^\prime_n)$ ist $M_{s, s^\prime}(\id_V)$ die Basiswechselmatrix.

\skript{3.18a}Mit Basen $s = (s_1,\ldots, s_n)$ und $s^\prime = (s^\prime_1, \ldots, s^\prime_n)$ von $V$ gilt für $v \in V$:\\ $\kappa_{s^\prime}(v) = (M_{s,s^\prime}(\id_V))^{-1} \kappa_s(v)$.

\skript{3.18b}Mit $\varphi : V \to W \in Hom(V,W)$, Basen $s = (s_1,\ldots, s_n)$ und $s^\prime = (s^\prime_1, \ldots, s^\prime_n)$ von $V$ und Basen $t = (t_1,\ldots, t_n)$ und $t^\prime = (t^\prime_1, \ldots, t^\prime_n)$ von $W$ gilt: \\
$M_{t^\prime, s^\prime}(\varphi) = (M_{t,t^\prime}(\id_W))^{-1} M_{t,s}(\varphi) M_{s,s^\prime}(\id_V)$

\subsection{Kardinalitäten von versch. Mengen}

$$\abs{\Sol(A,0)} \text{ für ein } A \in \Field_q^{m \times n} \text{ mit } \rk_{\Field_q} A = r \text{ ist } q^{n-r}$$
Da Elemente aus $\Sol(A,0)$ Vektoren aus $\Field_q^n$ sind, müssen wir $n$ Parameter aus $\Field_q$ wählen, von denen $r$ nicht frei sind.

$$\abs{\{ U \leq \Field_q^n | \dim_{\Field_q} U = k \}} = \binom{n}{k}_q$$
Der Gauß'sche Binomialkoeffizient. Ja, die Ergebnisse werden schnell groß.

$$\abs{\mathrm{Hom}(\Field_q^n,\Field_q^m)} = \abs{\Field_q^{m \times n}} = q^{m \cdot n}$$
Homomorphismen lassen sich als Matrizen darstellen und umgekehrt. Daher das ganze Matrixkalkül-Ding.

$$\abs{\GL_n(\Field_q)} = \prod_{i=0}^{n-1} q^n-q^i$$
$$\abs{\{A \in \Field_q^{m \times n} ~ | ~ \rk(A) = \min(m,n)\}} = \prod_{i=0}^{\min(m,n)-1} (q^{\max(m,n)}-q^i)$$
Oben müssen wir Zeilen oder Spalten linear unabhängig wählen, genau $\min(m,n)$ Stück mit Länge $\max(m,n)$. Die Erste ist dann bis auf die Null frei zu wählen, danach sollte jede folgende kein Vielfaches der Vorherigen sein.

\end{document}
