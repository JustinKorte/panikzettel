\documentclass[12pt,a4paper]{scrartcl}
\usepackage[utf8]{inputenc}
\usepackage{amsmath}
\usepackage{amsfonts}
\usepackage{amssymb}
\usepackage{tikz}
\usepackage{graphicx}
\usepackage{cite}
\usepackage[ngerman]{babel}
\usepackage[ngerman,num]{isodate}

%Def authors once
\newcommand{\authors}{Daniel Sous}

\usepackage[pdftitle={Entscheidungslehre - Panikzettel},
			pdfauthor={\authors},
			pdfsubject={In diesem Dokument befindet sich die inoffizielle Kurzzusammenfassung zur Vorlesung Entscheidungslehre - RWTH Aachen 2018/19.},
			pdfkeywords={Zusammenfassung, Entscheidungslehre, Kurz, Kurzzusammenfassung}]{hyperref}
\author{}
\title{Entscheidungslehre}
\subtitle{Panikzettel}
\author{\authors}

\monthyearsepgerman{}{}
\daymonthsepgerman{}{}
\date{\small Version 2 - \today}


\begin{document}
\maketitle

\begin{abstract}
	\textit{Hinweis:} Dies ist ein kurzer Überblick über die wichtigsten Formel und Aussagen, die in der Vorlesung \textit{Entscheidungslehre} \cite{vonNitzsch:skript} vorgestellt wurden. Die Zusammenfassung konzentriert sich vor allem auf Formeln und Rechnungen im Teil B \& C. Da es sich um eine Kurzzusammenfassung (aka Panikzettel) handelt, wird auf ausführliche Erklärungen und Herleitungen verzichtet.
\end{abstract}
Dieser Panikzettel ist Open Source.
Wir freuen uns über Anmerkungen und Verbesserungsvorschläge auf \url{https://git.rwth-aachen.de/philipp.schroer/panikzettel}.

\vspace{1cm}
\setcounter{tocdepth}{1}
\tableofcontents

\newpage
	
\section{Einführung}

\subsection{Bayes-Theorem}
Mit dem Bayes-Theorem lassen sich bedingte Wahrscheinlichkeiten ``umdrehen``:\\
Mit
\begin{align*}
\begin{array}{rrcll}
& P(A | B) & =  & \dfrac{P(A \cap B)}{P(B)} & |\cdot P(B) \\
\Leftrightarrow & P (A|B) \cdot P(B) & = & P(A \cap B)
\end{array}
\end{align*}
und 
\begin{align*}
\begin{array}{rrcll}
& P(B | A) & =  & \dfrac{P(A \cap B)}{P(A)} & |\cdot P(A) \\
\Leftrightarrow & P (B|A) \cdot P(A) & = & P(A \cap B)
\end{array}
\end{align*}
folgt
\begin{align*}
	P (A|B) \cdot P(B) = P(A \cap B) = P (B|A) \cdot P(A).
\end{align*}
Durch Umstellen folgt das \textbf{Bayes-Theorem}:
\begin{align*}
	P(A|B) = P(B|A) \cdot \dfrac{P(A)}{P(B)}
\end{align*}

\subsection{Standardnormalverteilung}
Transformation einer Normalverteilung\footnote{Auf weitere Verteilungen wie die Binomialverteilung, Exponentialverteilung, etc. wird hier verzichtet, da diese nicht auswendig zu lernen sind. Der Umgang mit diesen sollte jedoch bekannt sein.} mit Erwartungswert $ \mu $ und Standardabweichung $ \sigma $ (bzw. Varianz $ \sigma^2 $) in eine Standardnormalverteilung $ N(0,1) $ mit $ \mu = 0, \sigma = \sigma^2 = 1 $:
\begin{align*}
	Z &= \frac{X - \mu}{\sigma} \sim N(0,1)\\
	F(x) = P(X \leq x) &= P\left(\frac{X-\mu}{\sigma} \leq \frac{x-\mu}{\sigma}\right) = \phi \left(\frac{x-\mu}{\sigma}\right) = \phi(z)
\end{align*}

\newpage
\section{Präskriptive Entscheidungstheorie}

\subsection{Exponentielle Nutzenfunktion}
\begin{align*}
	u(x) := \begin{cases}
	\dfrac{1 - e ^{-c\frac{x-x^-}{x^+-x^-}}}{1-e^{-c}} & \text{für } c \neq 0\\\\
	\dfrac{x-x^-}{x^+-x^-} & \text{für } c = 0
	\end{cases} && \text{mit } c := -2\text{ln}\left(\frac{1}{p}-1\right)
\end{align*}
$ p $ lässt sich dabei durch Erfragen in folgendem indifferent Spiel ermitteln:
\begin{figure}[h]
	\centering
	\begin{tikzpicture}
		\node[](safe) at (-1.2,0) {$ \dfrac{x^-+x^+}{2} $};
		\node[](tilde) at (0,0) {$ \sim $};
		\node[](x+) at (2,0.5) {$ x^+ $};
		\node[](x-) at (2,-0.5) {$ x^- $};
		
		\path (0.5,0) edge node[auto,scale=0.7,inner sep=0,pos=0.6] {$ p $} (x+);
		\path (0.5,0) edge node[auto,scale=0.7,inner sep=0,swap,pos=0.8] {$ 1-p $} (x-);
	\end{tikzpicture}
\end{figure}\\
Es gilt außerdem
\begin{align*}
c\left\lbrace \begin{array}{cl}
>0 & \text{Risikoscheue}\\
=0 & \text{Risikoneutralität}\\
<0 & \text{Risikofreude}
\end{array}\right. .
\end{align*}

\subsection{Erwartungsnutzenmodell}
Eine Alternative $ a $ wird durch $ a := (p_1, a_1; \dots; p_n, a_n) $ definiert, wobei $ p_i $ für die Wahrscheinlichkeit und $ a_i $ für die Ausprägung im Zustand $ i\in\{1, \dots, n\} $. Für den Nutzen der Alternative $ a $ gilt nun
\begin{align*}
	EU(a) := \sum_{i=1}^{n} (p_i \cdot u(a_i))
\end{align*}
mit Nutzenfunktion $ u(x) $. Es wird also der Erwartungswert über die Nutzenwerte gebildet.

\subsection{Additive Nutzenfunktion bei Sicherheit}\label{sec:addUtility}
Bei Sicherheit lässt sich das additive Modell mit $ m $ Zielen und Alternativen $ a = (a_1, \dots, a_m) $ wie folgt definieren
\begin{align*}
	u(a) := \sum_{r=1}^{m} w_r\cdot u_r(a_r) && \text{mit } \sum_{r=1}^{m} w_r = 1 \text{ und } w_r > 0
\end{align*}

\subsection{Additive Nutzenfunktion bei Risiko}
Sei $ a_{ij} $ die Ausprägung der Alternative $ a $ im $ i $-ten Zustand und $ j $-ten Ziel, sowie $ p(s_i) $ die Wahrscheinlichkeit des Umweltzustands $ s_i $, dann gilt:
\begin{align*}
	EU(a) := \sum_{i=1}^{n} p(s_i) \cdot (w_1u_1(a_{i1}) + \dots + w_mu_m(a_{im}))
\end{align*}
Gewichte wie bei \hyperref[sec:addUtility]{\nameref{sec:addUtility}}

\subsection{Trade-Off Verfahren}
Sind die Zielgewichte $ w_1, \dots, w_r $ mit $ r\in\mathbb{N} $ gesucht und $ r-1 $ indifferente Präferenzen angegeben, kann das Trade-Off Verfahren angewendet werden, um die Gewichte zu bestimmen. Hier werden die indifferenten Präferenzen gleichgesetzt und nach einem beliebigen Gewicht umgeformt:
\begin{align*}
	w_1u_1(a_1) + \dots + w_ru_r(a_r) = w_1u_1(b_1) + \dots + w_ru_r(b_r)
\end{align*}
Für $ r=2 $ ergibt sich zum Beispiel:
\begin{align*}
	w_1 = \frac{u_2(a_2) - u_2(b_2)}{u_1(b_1) - u_1(a_1)} w_2
\end{align*}
Da gilt $ w_1 + \dots + w_r = 1 $ lassen sich nun die absoluten Werte der Gewichte berechnen.

\textit{Hinweis:} Falls das entstehende Gleichungssystem überbestimmt ist (d.h. es gibt mehr als $ r-1 $ indifferente Präferenzen), muss überprüft werden, ob eine gültige Lösung für alle Präferenzen besteht. Falls nicht, kann keine additive Nutzenfunktion verwendet werden.

\subsection{Dominanz bei unvollständiger Information}
\begin{itemize}
	\item Falls $ \text{min} \left[p(s_1) \cdot (u(a) - u(b)) + \dots + p(s_n) \cdot (u(a) - u(b)) \right] \geq 0$ gilt, dominiert Alternative $ a $ Alternative $ b $. Dabei sind die Wahrscheinlichkeiten $ p(s_1), \dots, p(s_n) $ jeweils aus dem gegebenen Intervall zu wählen, wobei $ p(s_1) + \dots + p(s_n) = 1 $ gelten muss.
	\item Falls nur die Rangfolge der Wahrscheinlichkeit gegeben ist, müssen folgende Punkte gelten, damit eine Dominanz vorliegt:
	\begin{enumerate}
		\item Die Ausprägung im wahrscheinlichsten Zustand muss besser sein als die aller anderen.
		\item Im Durchschnitt müssen die Ausprägungen am größten sein.
		\item Bei Kumulierung nach absteigender Wahrscheinlichkeit muss die dominierende Alternative stets größer oder gleich alle anderen Alternativen sein.
	\end{enumerate}
\end{itemize}

\newpage
\section{Deskriptive Entscheidungstheorie}
\subsection{Verzerrungen in der Informationswahrnehmung \cite{vonNitzsch:211553}}
\begin{itemize}
	\item \textit{Selektive Wahrnehmung:}\\
	Menschen beschränken ihre Wahrnehmung derart, dass Erwartungen eintreffen und eigene Entscheidungen als ``richtig`` erscheinen.
	\item \textit{Kontrast-Effekte:}\\
	Informationen, die mit einer im Kontrast stehenden Information präsentiert werden, worden oft überhöht wahrgenommen.
	\item \textit{Verfügbarkeitsheuristik:}\\
	Informationen, die im Kopf am leichtesten verfügbar sind, bestimmen das Entscheidungs- und Schätzverhalten, d.h. je verfügbarer ein Ereignis ist, desto größer ist seine subjektive Wahrscheinlichkeit.
	\item \textit{Mentale Kontenführung:}\\
	Menschen neigen dazu, alle Projekte, an denen sie arbeiten, isoliert in einem zugehörigen ``Mental Account`` zu verbuchen, wobei gegenseitige Abhängigkeiten zwischen diesen vernachlässigt werden.
	\item \textit{Verankerungsheuristik:}\\
	Personen werden bei Schätzungen von Wahrscheinlichkeiten von einem vorgegebenen Anker beeinflusst, wobei gegenseitige Anpassung unter Berücksichtigung weiterer Informationen zu gering ausfällt.
	\item \textit{Repräsentativitätsheuristik:}\\
	Neigung der Menschen, zu schnell in ein Schema-Denken zu verfalle und die Wahrscheinlichkeit von repräsentativen Ereignissen zu überschätzen.
\end{itemize}

\subsection{Wertefunktion}
Die Wertefunktion spiegelt den subjektiven Wert wieder, welcher für einen Gewinn ($ x \geq 0 $) oder einen Verlust ($ x < 0 $) wahrgenommen wird.
\begin{align*}
	v(x) := \begin{cases}
	\sqrt{x} & \text{für } x\geq 0,\\
	-\lambda\sqrt{-x} & \text{für } x < 0
	\end{cases}
\end{align*}
Ein Commitment drückt sich in dem Parameter $ \lambda $ bei Verlusten aus (\textit{Verlustaversion}). Dieser lässt Verluste für $ \lambda > 1 $ noch ``schlimmer`` wirken.

\subsection{Wahrscheinlichkeitsgewichtefunktion}
Menschen überschätzen geringe Wahrscheinlichkeiten und überbewerten die absolute Sicherheit. Die allgemeine Wahrscheinlichkeitsgewichtefunktion ist durch $ \pi_{\delta,\gamma}(p) $ gegeben:
\begin{align*}
	\pi_{\delta,\gamma}(p) := \frac{\delta\cdot p^\gamma}{\delta\cdot p^\gamma + (1-p)^\gamma}
\end{align*}
\begin{figure}[h]
	\centering
	\begin{tikzpicture}[scale=5]
		\draw[dotted,thin,step=0.2,gray] (0,0) grid (1,1);
		\draw[->] (0,0) -- (1,0) node[right] {100\%};
		\draw[->] (0,0) -- (0,1) node[above] {100\%};
		\draw[domain=0:1,smooth,samples=200,variable=\x,thick] plot ({\x},{(\x^0.5)/(\x^0.5 + (1-\x)^0.5)});
		\draw[domain=0:1,variable=\x,gray] plot ({\x}, {\x});
		\node[](zero) at (-0.05,-0.05) {0\%};
	\end{tikzpicture}
	\caption{Graph der Funktion $ \pi_{1,0.5}(p) $}
\end{figure}

\subsection{Risikoeinstellung}
Sei eine sichere Alternative $ A $ und ein Spiel $ B $ - mit der Wahrscheinlichkeit $ p_1 $ einen Betrag $ b_1 $ und $ p_2 $ einen Betrag $ b_2 $ zu gewinnen - gegeben. Des Weiteren sei eine Wertefunktion $ v(x) $ gegeben. Falls
\begin{align*}
	v(A) = p_1 \cdot v(b_1) + p_2 \cdot v(b_2)
\end{align*}
gilt, dann folgt
\begin{align*}
\begin{array}{ccll}
A & \succ & B\footnotemark & \Rightarrow \text{risikoscheue Einstellung}\\
A & \sim & B & \Rightarrow \text{risikoneutrale Einstellung}\\
A & \prec & B & \Rightarrow \text{risikofreudige Einstellung}\\
\end{array}
\end{align*}
\footnotetext{$ A \succ B $ steht dafür, dass Alternative $ A $ gegenüber Alternative $ B $ bevorzugt wird.}

\subsection{Risikoverhalten}
Um den Begriff des \textit{Risikoverhaltens} zu erläutern, muss zunächst der Begriff \textit{Risikoprämie} definiert werden. Dazu sei ein Spiel mit dem Erwartungswert $ EW $ und ein alternatives Sicherheitsäquivalent $ S\ddot{A} $ gegeben. Dann gilt für die Risikoprämie $ RP $
\begin{align*}
	RP := EW - S\ddot{A}.
\end{align*}
Wird das gegebene Spiel und das Sicherheitsäquivalent als indifferent angesehen, so lässt sich das Risikoverhalten ableiten:
\begin{align*}
	RP \left\lbrace \begin{array}{ccl}
	> & 0 & \text{Entscheider verhält sich risikoscheu}\\
	= & 0 & \text{Entscheider verhält sich risikoneutral}\\
	< & 0 & \text{Entscheider verhält sich risikfreudig}
	\end{array}\right.
\end{align*}
Das Risikoverhalten ist das beobachtbare Verhalten in Risikosituationen. Es bildet das Resultat aus Höhenpräferenzen und Risikoeinstellung.

\subsection{Discounted-Utility-Modell}
Der heutige Wert eines zukünftigen Ereignisses wird durch Abdiskontierung seines späteren Nutzens auf den heutigen Zeitpunkt abgebildet:
\begin{align*}
	DU(a) :=& \sum_{t=0}^{T} \left(\frac{1}{1+i}\right)^t u_t(a_t) \\
	=& \sum_{t=0}^{T} e^{-t\cdot \text{ln}(1+i)} u_t(a_t)
\end{align*}
Wobei
\begin{align*}
	u_t(a_t) &= \text{Nutzen des Ergebnisses } a_t \text{ im Zeitpunkt } t\\
	i &= \text{Diskontrate}
\end{align*}
\textit{Hinweis:} Zu beachten ist, dass alle Summanden wegfallen, die zu einem Zeitpunkt mit Nutzenfunktion $ u_t(a_t) = 0 $ gehören. D.h. bekommt man im \textit{12. Monat} einen Nutzen und in allen anderen Monaten nicht, fallen die Summanden mit $ t = 0, \dots, 11 $ weg. Es bleibt also nur noch $ DU(a) = \left(\frac{1}{1+i}\right)^{12} u_{12}(a_{12}) $.

\subsection{Hyperbolic-Discounted-Utility-Modell}
\begin{align*}
	HDU(a) :=& \sum_{t=0}^{T} \delta^\text{hyp}(t)u_t(a_t)\\
	=& \sum_{t=0}^{T} \left(\frac{1}{1+\alpha t}\right)^{\frac{\beta}{\alpha}} u_t(a_t)
\end{align*}

\newpage
\section{Wege zu einem besseren Entscheiden}
\subsection{Anwendungsfelder der Entscheidungslehre \cite{vonNitzsch:211553}}
\begin{enumerate}
	\item Verbesserung der Entscheidungsqualität
	\item Beeinflussung des Verhaltens Dritter zum eigenen Nutzen
	\item Beeinflussung der Verhalten Dritter zu deren Nutzen oder zum Nutzen der Gesellschaft (Nudging)
	\item Beeinflussung des eigenen Verhaltens (Selbstlenkung)
	\item Veränderung der Wahrnehmung zur Zufriedenheitssteigerung (Hedonic Framing)
	\item Erlangen eines eigenen Profits aus der Verhaltensprognose anderer
\end{enumerate}

\subsection{Entscheidungen im Team}
Um eine Entscheidung im Team treffen zu können, müssen zunächst die jeweiligen Stakeholder (Personen die ein begründetes Interesse an dem Projekt haben) identifiziert und eingeordnet werden. Dabei muss zwischen den folgenden zwei Aspekten unterschieden werden:
\begin{itemize}
	\item Relative Gewichtung der Ziele des  Stakeholders im Verhältnis der eigenen:
	
	Hier sollte man sich die Frage stellen, ob derjenige Stakeholder die gleichen Ziele verfolgt und sich damit für das Projekt engagieren wird.
	
	\item Ausmaß der instrumentellen Bedeutung des Stakeholders für die eigenen Ziele:
	
	In diesem Aspekt geht es darum, wie viel Einfluss der jeweilige Stakeholder auf die Entscheidung hat. ``Könnte der Stakeholder das Projekt stoppen, wenn er sich nicht genug einbezogen fühlt?\grqq
\end{itemize}
Die Auswertung dieser beiden Aspekte kann nun in einem Netzdiagramm dargestellt werden.

\subsection{Analytisches Vorgehen \cite{vonNitzsch:211553}}
Sowohl für Individual- als auch für Gruppenentscheidungen sollten die folgenden fünf Schritte für ein analytisches Vorgehen berücksichtigt werden:
\begin{enumerate}
	\item[(Z)] Festlegung des Zielkatalogs
	\item[(A)] Alternativen suche
	\item[(EF)] Identifikation der unsicheren Einflussfaktoren
	\item[(P)] Umwelt- und Wirkungsprognosen
	\item[(B)] Analytische Bewertung
\end{enumerate}

\subsection{Entscheidungsnavi}
Die 8 Menüpunkte des Entscheidungsnavis sind (Stand: 03.02.2019):
\begin{enumerate}
	\item Entscheidungsfrage
	\item Ziele
	\item Alternativen
	\item Unsicherheitsfaktoren
	\item Wirkungsprognosen
	\item Nutzenfunktion
	\item Zielgewichte
	\item Auswertung
	
\end{enumerate}

\newpage
% Generated by IEEEtran.bst, version: 1.14 (2015/08/26)
\begin{thebibliography}{1}
	\providecommand{\url}[1]{#1}
	\csname url@samestyle\endcsname
	\providecommand{\newblock}{\relax}
	\providecommand{\bibinfo}[2]{#2}
	\providecommand{\BIBentrySTDinterwordspacing}{\spaceskip=0pt\relax}
	\providecommand{\BIBentryALTinterwordstretchfactor}{4}
	\providecommand{\BIBentryALTinterwordspacing}{\spaceskip=\fontdimen2\font plus
		\BIBentryALTinterwordstretchfactor\fontdimen3\font minus
		\fontdimen4\font\relax}
	\providecommand{\BIBforeignlanguage}[2]{{%
			\expandafter\ifx\csname l@#1\endcsname\relax
			\typeout{** WARNING: IEEEtran.bst: No hyphenation pattern has been}%
			\typeout{** loaded for the language `#1'. Using the pattern for}%
			\typeout{** the default language instead.}%
			\else
			\language=\csname l@#1\endcsname
			\fi
			#2}}
	\providecommand{\BIBdecl}{\relax}
	\BIBdecl
	
	\bibitem{vonNitzsch:skript}
	\BIBentryALTinterwordspacing
	R.~von Nitzsch, \emph{{Entscheidungslehre : Wie Menschen entscheiden und wie
			sie entscheiden sollten}}.\hskip 1em plus 0.5em minus 0.4em\relax Aachen:
	Mainz, 2017. [Online]. Available: \url{http://d-nb.info/1060414678}
	\BIBentrySTDinterwordspacing
	
	\bibitem{vonNitzsch:211553}
	\BIBentryALTinterwordspacing
	R.~von Nitzsch, S.~Schiffer, and P.~von Thunen, \emph{Übungen zur
		{E}ntscheidungslehre : Übungsaufgaben und alte {K}lausuraufgaben mit
		{M}usterlösungen, {F}allstudien, {F}achbegriffe; 8. überarbeitete.
		[{A}ufl.]}.\hskip 1em plus 0.5em minus 0.4em\relax Aachen: Mainz, 2017.
	[Online]. Available: \url{http://publications.rwth-aachen.de/record/211553}
	\BIBentrySTDinterwordspacing
	
\end{thebibliography}
	
\end{document}